\documentclass[xcolor=pdftex,romanian,colorlinks]{beamer}

\usepackage[export]{adjustbox}
\usepackage{../tslides}
\usepackage[all]{xy}
\usepackage{pgfplots}
\usepackage{flowchart}
\usepackage{comment}
\usetikzlibrary{arrows,positioning,calc}
\lstset{language=Haskell}
\lstset{escapeinside={(*@}{@*)}}

\AtBeginSection[]{
  \begin{frame}
  \vfill
  \centering
  \begin{beamercolorbox}[sep=8pt,center,shadow=true,rounded=true]{title}
    \usebeamerfont{title}\insertsectionhead\par%
  \end{beamercolorbox}
  \vfill
  \end{frame}
}


\title[PD---Monade]{Programare declarativă\thanks{bazat pe cursul \emph{Informatics 1: Functional Programming} de la \emph{University of Edinburgh}}}

\subtitle{De la functori la monade}

\begin{document}
\begin{frame}
  \titlepage
\end{frame}

\section{Functori}

\subsection{Matematică}

\begin{frame}{Categorii}

\begin{block}{Intuiție}
O categorie este dată de \structure{obiecte} și \alert{săgeți} între ele
\end{block}
\begin{block}{Exemple}
\begin{description}
\item[$\mathbb Set$:] \structure{mulțimi} și \alert{funcții}
\item[$\mathbb Poset$:] \structure{Mulțimi parțial ordonate} și \alert{funcții monotone}
\item[$\mathbb Mon$:] \structure{Monoizi} și \alert{morfisme de monoizi}
\item[$\mathbb Top$:] \structure{spații topologice} și \alert{morfisme de spații topologice}
\item[$\mathbb Hask$:] \structure{tipuri} și \alert{funcții} în Haskell
\end{description}
\end{block}
\end{frame}

\begin{frame}{Functori}
\begin{block}{Intuiție}
Un functor este o transformare între două categorii
\begin{itemize}
\item Duce obiecte în obiecte și săgeți în săgeți în mod corespunzător
\item Compatibilă cu compunerea și cu identitățile
\end{itemize}
\end{block}

\begin{block}{Exemple}
\begin{itemize}
\item Functorii „uituci” de la $\mathbb{P}oset$ sau $\mathbb{M}on$ în $\mathbb{S}et$
\item Functorul „liber” de la $\mathbb{S}et$ în $\mathbb{M}on$
\begin{itemize}
\item Duce o mulțime $\Sigma$ în monoidul cuvintelor peste alfabetul $\Sigma$: $(\Sigma^\ast, \cdot, \lambda)$
\item O funcție între alfabete se extinde în mod unic pe cuvinte
\end{itemize}
\end{itemize}
\end{block}
\end{frame}

\begin{frame}{Endofunctori}
\begin{block}{Definiție}
Un endofunctor este un functor de la o categorie la ea însăși.
\end{block}

\begin{block}{Exemple în $\mathbb{S}et$}
\begin{itemize}
\item Functorul părților
\begin{itemize}
\item duce o mulțime $A$ în $\mathcal{P}(A)$, mulțimea părților lui $A$
\item duce $f : A \rightarrow B$ în $\mathcal{P}(f) : \mathcal{P}(A) \rightarrow \mathcal{P}(B)$ care dă de imaginea prin $f$ a oricărei submulțimi a lui $A$
\end{itemize}
\item Functorul funcțiilor de sursă $S$ dată 
\begin{itemize}
\item duce o mulțime $A$ în $A^S$, mulțimea funcțiilor de la $S$ la $A$
\item duce $f : A \rightarrow B$ în $f^S : A^S \rightarrow B^S$ dată de 
$f^S(h) = h \circ f$ pentru orice $h : S \rightarrow A$
\end{itemize}

\end{itemize}
\end{block}
\end{frame}

\subsection{În Haskell}

\begin{frame}[fragile]{Constructori de tipuri (cu un parametru)}

\begin{block}{Observație}
Un constructor de tipuri asociază fiecărul tip un tip nou bazat pe acel tip
\end{block}

\begin{block}{Exemple}
\begin{itemize}
\item \lstinline$Maybe$ asociază unui tip tipul valorilor parțiale de acel tip
\item \lstinline$[]$ asociază unui tip tipul listelor cu elemente de acel tip
\item \lstinline$IO$ asociază unui tip tipul computațiilor I/O cu rezultate de acel tip
\end{itemize}
\end{block}

\begin{itemize}
\item Acești constructori acționează ca niște endofunctori pe categoria $\mathbb{H}ask$

 \ldots dar doar pe obiecte (tipuri) 
 
\item Putem defini și acțiunea lor pe săgeți (funcții)?
\end{itemize}
\end{frame}


\begin{frame}[fragile]{Clasa de tipuri Functor}

\begin{block}{Definiție}
\vspace{-1ex}
\begin{asciihs}
class Functor f where
  fmap :: (a -> b) -> f a -> f b
\end{asciihs}
\vspace{-1ex}
Înzestrează un constructor de tipuri cu o modalitate de a asocia funcțiilor între argumente funcții între tipurile construite folsoind acele argumente
\end{block}

\begin{block}{Exemplu: Functorul listelor}
\vspace{-1ex}
\begin{asciihs}
instance Functor [] where
  fmap = map
\end{asciihs}
\vspace{-1ex}
\lstinline$fmap f$ e funcția care aplică f fiecărui element al listei argument

\begin{itemize}
\item De comparat cu functorul $\mathcal P$ peste $\mathbb{S}et$
\end{itemize}
\end{block}
\end{frame}

\begin{frame}[fragile]{Clasa de tipuri Functor}{Functorul Maybe}
\begin{block}{Definiție}
\vspace{-1ex}
\begin{asciihs}
class Functor f where
  fmap :: (a -> b) -> f a -> f b
\end{asciihs}
\end{block}


\begin{block}{Functorul Maybe}
\vspace{-1ex}
\begin{asciihs}
instance Functor Maybe where
  fmap _ Nothing  = Nothing
  fmap f (Just x) = Just (f x)
\end{asciihs}
\end{block}

\begin{block}{Abstractizare:}
\begin{itemize}
\item un constructor de tipuri crează o „cutie” abstractă care ține valori
\item fmap „ridică” o funcție între valori la o funcție între „cutii”
\end{itemize}
\end{block}
\end{frame}


\begin{frame}[fragile]{Clasa de tipuri Functor}{Functorul funcțiilor de sursă s}
\begin{block}{Definiție}
\vspace{-1ex}
\begin{asciihs}
class Functor f where
  fmap :: (a -> b) -> f a -> f b
\end{asciihs}
\end{block}

\begin{block}{Functorul funcțiilor de sursă s}
\vspace{-1ex}
\begin{asciihs}
instance Functor (->) s where
  fmap f h  = f . h
\end{asciihs}
\vspace{-1ex}
De comparat cu versiunea simiară în $\mathbb{S}et$ a acestui functor
\end{block}
\end{frame}

\begin{frame}[fragile]{fmap --- Exemple de folosire}
\vspace{-1ex}
\begin{asciihs}
Prelude Data.Functor> fmap length (Just "something")
Just 9
Prelude Data.Functor> length <$> Just "something"
Just 9
Prelude Data.Functor> (++" else") <$> Just "something"
Just "something else"
Prelude Data.Functor> fmap (++" else") ["something","anything"]
["something else","anything else"]
Prelude Data.Functor> length <$> ["something","anything"]
[9,8]
Prelude Data.Functor> length <$> Nothing
Nothing
Prelude Data.Functor> fmap length []
[]
Prelude Data.Functor> length <$> getLine
something eIse
14
\end{asciihs}
%$
\end{frame}

\begin{frame}[fragile]{Clasa de tipuri Functor}{Proprietăți}
Instanțierile clasei de tipuri Functor trebuie să satisfacă următoarele properietăți:
\begin{itemize}
\item Conservarea identităților: \lstinline$fmap id  ==  id$
\item Compatibilitatea cu compunerea: \lstinline$fmap (f . g)  ==  fmap f . fmap g$
\end{itemize}

\begin{block}{Exercițiu}
Instanțele prezentate mai sus satisfac aceste proprietăți.
\end{block}
\end{frame}

\begin{frame}[fragile]{Clasa de tipuri Functor}{Limitări}
\begin{itemize}
\item fmap „ridică” funcțiile cu un singur argument

\lstinline$fmap :: (a -> b) -> f a -> f b$
\item Dar ce facem cu funcțiile cu mai multe argumente, e.g., operatorii?

\lstinline$(+) :: Int -> (Int -> Int)$

\lstinline$fmap (+) :: f Int -> f (Int -> Int)$

\lstinline$fmap (+) (Just 7) == Just (7+) :: Maybe (Int -> Int)$

Cum folosesc mai departe \lstinline$Just (7+)$?
\end{itemize}
\end{frame}

\section{Functori aplicativi}

\begin{frame}[fragile]{Motivație}
\begin{asciihs}
type Name = String
data Employee = Employee { name    :: Name
                         , phone   :: String }
                deriving Show
\end{asciihs}                

\begin{asciihs}
Employee :: Name -> String -> Employee
\end{asciihs}

\begin{itemize}
\item Putem „ridica” această funcție în subcategoria Maybe?

\lstinline$Maybe Name -> Maybe String -> Maybe Employee$

\item Dar în subcategoria listelor?

\lstinline$[Name] -> [String] -> [Employee]$

\item Dar în subcategoria funcțiilor de sursă $s$?

\lstinline$(s -> Name) -> (s -> String) -> (s -> Employee)$

\item Putem rezolva problema în general pentru orice instanță a lui Functor?
\end{itemize}
\end{frame}

\begin{frame}[fragile]{Generalizare și impas}
\begin{block}{Generalizare}
\begin{itemize}
\item Avem \lstinline$h :: a -> b -> c$
\item Vrem ceva gen \lstinline$fmap2 h :: f a -> f b -> f c$
\end{itemize}
\end{block}

\begin{block}{Impas}
\begin{itemize}
\item Chiar dacă presupunem că f e instanță a lui Functor, nu putem defini 

\lstinline$fmap2 :: Functor f => (a -> b -> c) -> (f a -> f b -> f c)$
\item Deoarece \lstinline$fmap h :: f a -> f (b -> c)$, de unde
\item \lstinline$fmap h fa :: f (b -> c)$ pentru \lstinline$fa :: f a$
\item Și nu avem un mecanism de a \structure{aplica} \lstinline$fmap h fa$ unui  \lstinline$fb :: f b$
\end{itemize}
\end{block}
\end{frame}

\begin{frame}[fragile]{Clasa de tipuri Applicative}
\begin{block}{Definitie}
\vspace{-1ex}
\begin{asciihs}
class Functor f => Applicative f where
  pure  :: a -> f a
  (<*>) :: f (a -> b) -> f a -> f b
\end{asciihs}
\vspace{-1ex}
Un functor este aplicativ dacă definește 
\begin{itemize}
\item o modalitate de a aplica „cutiile” continând funcții la „cutiile” conținând valori (operatorul \lstinline$<*>$)
\item o modalitate de a vedea orice valoare ca o „cutie” conținând acea valoare (funcția „pure”)
\end{itemize}
\end{block}

\begin{block}{fmap* pentru functori aplicativi}
\vspace{-1ex}
\begin{asciihs}
fmap2 :: Applicative f =>(a->b->c)->(f a->f b->f c)
fmap2 h fa fb = h <$> fa <*> fb

fmap3 :: Applicative f =>(a->b->c->d)->(f a->f b->f c->f d)
fmap3 h fa fb fc = h <$> fa <*> fb <*> fc
\end{asciihs}
\vspace{-1ex}
\end{block}
\end{frame}

\begin{frame}[fragile]{Exemplu: Maybe ca functor aplicativ}
\begin{block}{Instanțiere}
\vspace{-1.5ex}
\begin{asciihs}
instance Applicative Maybe where
  pure              = Just
  Nothing <*> _     = Nothing
  _ <*> Nothing     = Nothing
  Just f <*> Just x = Just (f x)
\end{asciihs}
\vspace{-1ex}
\end{block}
\begin{block}{Exemplu de folosire}
\vspace{-1.5ex}
\begin{asciihs}
m_name1, m_name2 :: Maybe Name
m_name1 = Nothing                 m_name2 = Just "Brent"
m_phone1, m_phone2 :: Maybe String
m_phone1 = Nothing                m_phone2 = Just "555-1234"

ex01 = Employee <$> m_name1 <*> m_phone1
ex02 = Employee <$> m_name1 <*> m_phone2
ex03 = Employee <$> m_name2 <*> m_phone1
ex04 = Employee <$> m_name2 <*> m_phone2
\end{asciihs}
\end{block}
\end{frame}

\begin{frame}[fragile]{Exemplu: Liste ca functor aplicativ}
\begin{block}{Instanțiere}
\vspace{-1.5ex}
\begin{asciihs}
instance Applicative [] where
  pure x            = [x]
  fs <*> xs         = [ f x | f <- fs, x <- xs ]
\end{asciihs}
\vspace{-1ex}
\end{block}
\begin{block}{Exemplu de folosire}
\vspace{-1.5ex}
\begin{asciihs}
*Main> import Control.Applicative
*Main Control.Applicative> Employee <$> ["John Smith",      "Johann Schultz"] <*> ["123-4567","555-1234"]
[Employee {name = "John Smith", phone = "123-4567"},
 Employee {name = "John Smith", phone = "555-1234"},
 Employee {name = "Johann Schultz", phone = "123-4567"},
 Employee {name = "Johann Schultz", phone = "555-1234"}]
\end{asciihs}
%$
\end{block}
\end{frame}

\begin{frame}[fragile]{Functori aplicativi}{Proprietăți}
\begin{description}
\item[Functorialitate]
    \lstinline$fmap f x = pure f <*> x$
\item[Identitate]

    \lstinline$pure id <*> v = v$

\item[Compunere]

    \lstinline$pure (.) <*> u <*> v <*> w = u <*> (v <*> w)$
    
    

\item[Homomorfism]

    \lstinline$pure f <*> pure x = pure (f x)$

\item[Interschimbare]

    \lstinline"u <*> pure y = pure ($ y) <*> u"
    
\end{description}

\begin{block}{Exercițiu}
Demonstrați că proprietățile enunțate mai sus țin pentru instanțele definite.
\end{block}
\end{frame}


\section{Monade}
\begin{frame}[fragile]{Motivație}

\begin{itemize}
\item Functorii aplicativi pot descrie și compune/secvenția computații \ldots

\item \ldots dar computațiile obținute au o structură rigidă

E greu să folosim rezultate intermediare pentru a controla computațiile următoare

\item Putem introduce o modalitate de a decide cum să continuăm o computație bazat pe rezultatul computației precedente?

\lstinline$(>>=) :: Applicative f => f a -> (a -> f b) -> f b$
\end{itemize}

\begin{block}{Impas}
\begin{itemize}
\item Fie \lstinline$fa :: f a$  și \lstinline$k :: a -> f b$
\item Pentru a putea aplica k lui fa, putem încerca să îl „ridicăm” pe k folosind fmap: \hfill \lstinline$fmap k :: f a -> f f b$
\item Acum putem aplica \lstinline$fmap k$ lui \lstinline$fa$ și obținem
\hfill \lstinline$fmap k fa :: f f b$

\item Cum „comprimăm” computația \lstinline$fmap k fa$ la tipul \lstinline$f b$?
\end{itemize}
\end{block}
\end{frame}

\begin{frame}{Monadă}
\begin{block}
{Definiție (Teoria categoriilor)}
O monadă este un tuplu $(\eta, T, \mu)$ unde $T : \mathbb{C} \rightarrow \mathbb{C}$ este un endofunctor, iar $\eta_X : X \Rightarrow T(X)$ și $\mu_X : T(T(X)) \Rightarrow T(X)$ sunt familii de funcții indexate de obiectele lui $\mathbb{C}$ cu proprietățile:
\begin{description}
\item[Naturalitate:] $T(f) \circ \eta_X = \eta_Y \circ f$
și $T(f)\circ \mu_X = \mu_Y \circ T(T(f))$

pentru orice săgeată $f : X \rightarrow Y$

\item[Asociativitate:] $\mu_X \circ T(\mu_X) = \mu_X \circ \mu_{T(X)}$

\item[Identitate:] $ \mu_X \circ T(\eta_X) = \mu_X\circ \eta_{T(X)}$

\end{description}
\end{block}

\begin{block}{În Haskell T este un functor aplicativ, $\eta_X = \texttt{pure :: Applicative t => a -> t a}$, $\nu_X = \texttt{join :: Applicative t => t t a -> t a} $}
\end{block}
\end{frame}


\begin{frame}{Monadă în Haskell}{Proprietăți}
\begin{description}
\item[Naturalitate:] \lstinline$fmap f . pure == pure . f$
și \lstinline$fmap f . join == join . fmap (fmap f)$

pentru orice \lstinline$f :: a -> b$

\item[Asociativitate:] \lstinline$join . fmap join == join . join$

\item[Identitate:] \lstinline$join . fmap pure == join . pure$
\end{description}

\begin{block}{Bind vs Join}
\begin{itemize}

\item Bind în funcție de join:

 \lstinline"ma >>= k == join $ fmap k ma"

\item Join în funcție de bind:

 \lstinline$join mma == mma >>= id$

\end{itemize}
\end{block}
\end{frame}

\begin{frame}[fragile]{Clasa de tipuri Monad}
\begin{block}{Definiție}
\begin{asciihs}
class Applicative m => Monad m where
    (>>=)       :: m a -> (a -> m b) -> m b
\end{asciihs}
\end{block}
\begin{block}{Exemplu---Monada Maybe}
\begin{asciihs}

instance Monad Maybe where
    (Just a) >>= k = k a
    Nothing  >>= _ = Nothing
\end{asciihs}
\end{block}

\end{frame}

\begin{frame}[fragile]{Clasa de tipuri Monad}{Monada listelor}
\begin{block}{Definiție}
\begin{asciihs}
class Applicative m => Monad m where
    (>>=)       :: m a -> (a -> m b) -> m b
\end{asciihs}
\end{block}
\begin{block}{Exemplu---Monada listelor}
\begin{asciihs}
instance Monad [] where
  ma >>= k = join $ fmap k ma
    where
      join = concat  
\end{asciihs}
%$
\end{block}
\end{frame}


\section{Evaluare cu efecte laterale}

\subsection{Sintaxă abstractă}

\begin{frame}[fragile]{Lambda calcul cu întregi}
{Sintaxă}
\begin{asciihs}
type Name = String
data Term
  = Var Name
  | Con Int
  | Add Term Term
  | Lam Name Term
  | App Term Term
\end{asciihs}
\end{frame}


\begin{frame}[fragile]
{Valori și medii de evaluare}
\begin{asciihs}
data Value m
  = Wrong
  | Num Int
  | Fun (Value m -> m (Value m))
  
instance Show (Value m) where
  show Wrong = "<wrong>"
  show (Num i) = show i
  show (Fun f) = "<function>"
\end{asciihs}
\end{frame}

\begin{frame}[fragile]
{Mediul de variabile}
\begin{asciihs}
type Environment m = [(Name,Value m)]

lookUp :: Monad m => Name -> Environment m -> m (Value m)
lookUp _ [] 
  = return Wrong
lookUp x ((y,b):e)
  = if x == y then return b else lookUp x e
\end{asciihs}
\end{frame}

\begin{frame}[fragile]
{Evaluare}{Variabile și valori}
\begin{asciihs}
interp :: Monad m => Term -> Environment m -> m (Value m)
interp (Var x) e   = lookUp x e
interp (Con i) e   = return (Num i)
interp (Lam x v) e = return (Fun (\a -> interp v ((x,a):e)))
\end{asciihs}
\end{frame}



\begin{frame}[fragile]
{Evaluare}{Adunare}
\begin{asciihs}
interp (Add u v) e = 
  do {
    a <- interp u e;
    b <- interp v e;
    add a b
  }

add :: (Monad m) => Value m -> Value m -> m (Value m)
add (Num i) (Num j) = return (Num (i+j))
add _       _       = return Wrong
\end{asciihs}
\end{frame}

\begin{frame}[fragile]
{Evaluare}{Aplicarea funcțiilor}
\begin{asciihs}
interp (App t u) e = 
  do {
    f <- interp t e;
    a <- interp u e;
    apply f a
  }


apply :: (Monad m) => Value m -> Value m -> m (Value m)
apply (Fun k) a = k a
apply _ _ = return Wrong
\end{asciihs}
\end{frame}


\begin{frame}[fragile]{Interpretare în monada Identitate}
\begin{asciihs}
newtype Identity a = Id a
instance Monad Identity where
  return a   = Id a
  Id a >>= k = k a

instance (Show a) => Show (Identity a) where
  show (Id a) = show a

evalId :: Term -> String
evalId t = show ( interp t ([]::Environment Identity) )
\end{asciihs}
\begin{itemize}
\item Obținem interpretorul standard discutat în cursurile trecute
\end{itemize}
\end{frame}


\begin{frame}[fragile]{Interpretare în monada Eroare}
\begin{asciihs}
data Error a = Success a | Error String
  (deriving Show)

instance Monad Error where
  return a = Success a
  fail   s = Error s
  (Success a) >>= k = k a
  (Error s)   >>= k = Error s
  
  
lookUp x [] = fail ("Unbound variable " ++ x)
add a b   = fail ("Should be numbers: " ++ show a ++ 
                                    "," ++ show b)
add f _   = fail ("Should be function: " ++ show f)
\end{asciihs}
\end{frame}

%\begin{frame}[fragile]{Interpretare în monada Stare}
%\begin{asciihs}
%data MState state a = MState (state -> (a,state))
%\end{asciihs}
%\end{frame}

\end{document}



