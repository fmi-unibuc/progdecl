\documentclass[addpoints,12pt,a4paper]{exam}
%,answers
\usepackage[none]{hyphenat} 
\usepackage{fullpage}
\usepackage{../tdefinition}
\pagestyle{empty}
%\header{Prenumele, numele și grupa: \enspace\hrulefill}{}{Subiect \thequestion}
\extrawidth{0.5in}
\newcommand{\bs}{\char`\\} \newcommand{\bt}{\char`\`}
\newcommand{\us}{\char`\_}

\begin{document}

\begin{center}
Programare declarativă---Examen scris \hfill  13 iunie 2016
\end{center}

\pointpoints{punct}{puncte}

\begin{questions}

\question[6] 
Recursivitate, descrieri de liste, map/filter/fold
\begin{parts}
\part[2] 
Să se scrie o funcție care dată fiind o listă de elemente de tip a și un predicat (funcție care ia elemente de tip a și întoarce o valoare de adevăr) întoarce o valoare opțională (Maybe) reprezentând primul element din listă care satisface predicatul dat.
\begin{asciihs}
  getFirst :: [a] -> (a -> Bool) -> Maybe a
  getFirst [1,3,2,5,7] even = Just 2
  getFirst [1,3,5,7] even = Nothing
  getFirst "0123456789abcdef" isLetter = Just 'a'
\end{asciihs}

Definiția poate folosi funcții elementare și recursie, dar nu descrieri de liste sau alte funcții din biblioteci.
\part[2]
Să se definească lista tuturor elementelor șirului recursiv definit de $a_0 = 1$, $a_1 = 2$, $a_{n+2} = a_{n+1} + 2*a_n + 1$.
\begin{asciihs}
  str :: [Integer]
  take 10 str = [1,2,5,10,21,42,85,170,341,682]
  str !! 100 = 1690200800304305868662270940501 
\end{asciihs}
Definiția poate folosi funcții elementare, descrieri de liste și funcții din biblioteci (e.g., zip).

\part[2] Să se scrie o funcție care primește ca parametru un șir de caractere și afișează valoarea corespunzătoare acestui șir obținută adunând valorile corespunzătoare literelor care au un echivalent în sistemul de numeratie roman.
Observații: (1) Nu se face distincție între litere mari și mici; (2) litera 'u'	 se identifica cu 'v' și litera 'j' se identifică cu 'i'; (3) I - 1, V - 2, X - 10, L - 50, C - 100, D - 500, M - 1000
\begin{asciihs}
  romanSum :: String -> Int
  romanSum "Marcus Ulpius Traianus" = 1172
  romanSum "Lucius Domitius Ahenobarbus" = 1673
  romanSum "Gaius Julius Caesar" = 167
\end{asciihs}
Definitia poate folosi una sau mai multe din funcțiile:
\begin{asciihs}
  map :: (a -> b) -> [a] -> [b]
  filter :: (a -> Bool) -> [a] -> [a]
  foldr :: (a -> b -> b) -> b -> [a] -> b
\end{asciihs}
Puteți folosi și funcții elementare, dar nu recursie, descrieri de liste sau alte funcții din biblioteci.
\end{parts}
\newpage
\question[3] 
Introducem un tip de date ce reprezinta o expresie booleana formată din literali / variabile (Lit), negație (Not), conjuncție (And), disjuncție (Or) și implicație (:->:), in care conjuncțiile si disjuncțiile au un număr arbitrar de termeni. 
\begin{asciihs}
data Exp = Lit String
         | Not Exp
         | And [Exp]
         | Or [Exp]
         | Exp :->: Exp
  deriving (Show)
\end{asciihs}

Un atom este fie un literal (Lit), fie negația unui literal. O expresie este în formă normală disjunctivă dacă este o disjuncție de conjuncții de atomi.

Să se scrie o funcție care dată fiind o expresie are ca rezultat forma normală disjunctivă a acelei expresii.

\begin{asciihs}
   dnf :: Exp -> Exp
   dnf ((Lit "a" :->: Lit "b") :->: Lit "c")
     = Or [And [Lit "a",Not (Lit "b")],And [Lit "c"]]
   dnf (Lit "a" :->: (Lit "b" :->: Lit "c"))
     = Or [And [Not (Lit "b")],And [Lit "c"],And [Not (Lit "a")]]
\end{asciihs}

Indicație---puteți folosi următoarele identități:
\begin{itemize}
\item $a \rightarrow b \equiv \neg a \vee b$
\item $\neg \neg a = a$
\item $\neg\bigwedge (a_1,\ldots,a_n) = \bigvee(\neg a_1,\ldots,\neg a_n)$
\item $\neg\bigvee (a_1,\ldots,a_n) = \bigwedge(\neg a_1,\ldots,\neg a_n)$
\item $\bigwedge{\left(a_1,\ldots,a_i,\bigwedge (b_1,\ldots,b_m),a_{i+1},\ldots,a_n\right)}
= \bigwedge (a_1,\ldots,a_n,b_1,\ldots,b_m)$
\item $\bigvee{\left(a_1,\ldots,a_i,\bigvee (b_1,\ldots,b_m),a_{i+1},\ldots,a_n\right)}
= \bigvee (a_1,\ldots,a_n,b_1,\ldots,b_m)$
\item $\bigwedge{\left(a_1,\ldots,a_i,\bigvee (b_1,\ldots,b_m),a_{i+1},\ldots,a_n\right)}
= \bigvee{\left(\bigwedge(b_1,a_1,\ldots,a_n),\ldots,\bigwedge(b_m,\ldots,a_n)\right)}$
\end{itemize}
Indicatie 2 --- Puteti folosi funcția getFirst de la Subiectul 1 pentru a căuta o expresie And sau Or într-o listă de expresii.

\end{questions}
Se acordă un punct din oficiu.
\end{document}

Să se scrie o funcție care dată fiind o listă de numere întregi calculează al doilea cel mai mic număr din listă.  Deoarece e posibil ca rezultatul să nu poată fi calculat, codomeniul funcției va fi de tip Opțiune (Maybe).
\begin{asciihs}
  second :: [Int] -> Maybe Int 
  second [3,2,5,1,6,8,4,6] = Just 2
  second [3,2,5,1,6,0,8,4,6] = Just 1
  second [1, 1] = Just 1
  second [0] = Nothing
\end{asciihs}
