\documentclass[handout,xcolor=pdftex,romanian,colorlinks]{beamer}

\usepackage[export]{adjustbox}
\usepackage{../tslides}
\usepackage[all]{xy}
\usepackage{pgfplots}
\usepackage{flowchart}
\usetikzlibrary{arrows,positioning,calc}
\lstset{language=Haskell}
\lstset{escapeinside={(*@}{@*)}}

\AtBeginSection[]{
  \begin{frame}
  \vfill
  \centering
  \begin{beamercolorbox}[sep=8pt,center,shadow=true,rounded=true]{title}
    \usebeamerfont{title}\insertsectionhead\par%
  \end{beamercolorbox}
  \vfill
  \end{frame}
}


\title[PD---Funcții de ordin înalt]{Programare declarativă}
\subtitle{Map, Filter, Fold}
\begin{document}
\begin{frame}
  \titlepage
\end{frame}

\section{Map (Transformarea fiecărui element dintr-o listă)}

\subsection{Exemple}

\begin{frame}[fragile]{Pătrate}
\begin{block}{}
Definiți o funcție care pentru o listă de numere întregi dată  ridică la pătrat fiecare element din listă.
\begin{asciihs}
*Main> squares [1,-2,3]
[1,4,9]
\end{asciihs}
\end{block}
\begin{block}{Soluție \structure{descriptivă}}
\begin{asciihs}
squares :: [Int] -> [Int]
squares xs = [ x * x | x <- xs ]
\end{asciihs}
\end{block}
\begin{block}{Soluție \structure{recursivă}}
\begin{asciihs}
squares :: [Int] -> [Int]
squares []     = []
squares (x:xs) = x*x : squares xs
\end{asciihs}
\end{block}
\end{frame}

\begin{frame}[fragile]
{Coduri ASCII}
\begin{block}{}
Transformați un șir de caractere în lista codurilor ASCII ale caracterelor.
\begin{asciihs}
*Main> ords "a2c3"
[97,50,99,51]
\end{asciihs}
\end{block}
\begin{block}{Soluție \structure{descriptivă}}
\begin{asciihs}
ords :: [Char] -> [Int]
ords xs = [ ord x | x <- xs ]
\end{asciihs}
\end{block}
\begin{block}{Soluție \structure{recursivă}}
\begin{asciihs}
ords :: [Char] -> [Int]
ords []     = []
ords (x:xs) = ord x : ords xs
\end{asciihs}
\end{block}
\end{frame}

\subsection{Funcția map}

\begin{frame}[fragile]{Funcția \lstinline$map$}{Definiție}
\begin{block}{}
Date fiind o funcție de transformare și o listă, aplicați funcția fiecărui element al unei liste date.
\end{block}
\begin{block}{Soluție \structure{descriptivă}}
\begin{asciihs}
map :: (a -> b) -> [a] -> [b]
map f xs = [ f x | x <- xs ]
\end{asciihs}
\end{block}
\begin{block}{Soluție \structure{recursivă}}
\begin{asciihs}
map :: (a -> b) -> [a] -> [b]
map f []     = []
map f (x:xs) = f x : map f xs
\end{asciihs}
\end{block}
\end{frame}


\begin{frame}[fragile]{Exemplu --- Pătrate}
\begin{block}{Soluție \structure{descriptivă}}
\begin{asciihs}
squares :: [Int] -> [Int]
squares xs = [ x * x | x <- xs ]
\end{asciihs}
\end{block}
\begin{block}{Soluție \structure{recursivă}}
\begin{asciihs}
squares :: [Int] -> [Int]
squares []     = []
squares (x:xs) = x*x : squares xs
\end{asciihs}
\end{block}
\begin{block}{Soluție folosind \lstinline$map$}
\begin{asciihs}
squares :: [Int] -> [Int]
squares xs = map sqr xs
  where sqr x = x * x
\end{asciihs}
\end{block}
\end{frame}

\begin{frame}[fragile]{Map în acțiune}{Varianta descriptivă}
\begin{minipage}{.49\columnwidth}%
\begin{asciihs}
squares :: [Int] -> [Int]
squares xs = map sqr xs
  where sqr x = x * x
\end{asciihs}
\end{minipage}
\begin{minipage}{.49\columnwidth}
\begin{asciihs}
map :: (a -> b) -> [a] -> [b]
map f xs = [ f x | x <- xs ]
\end{asciihs}
\end{minipage}

\vfill
squares [1,2,3]

\medskip
= map sqr [1,2,3]

\medskip
\onslide<2->
= [ sqr x | x <- [1,2,3]]

\medskip
\onslide<3->
= [ sqr 1 ] ++ [ sqr 2 ] ++ [ sqr 3 ]

\medskip
= [ 1, 4, 9 ]

\vfill\
\end{frame}


\begin{frame}[fragile]{Map în acțiune}{Varianta recursivă}
\begin{minipage}{.49\columnwidth}%
\begin{asciihs}
squares :: [Int] -> [Int]
squares xs = map sqr xs
  where sqr x = x * x
\end{asciihs}
\end{minipage}
\begin{minipage}{.49\columnwidth}
\begin{asciihs}
map :: (a -> b) -> [a] -> [b]
map f []     = []
map f (x:xs) = f x : map f xs
\end{asciihs}
\end{minipage}

\vfill
squares [1,2,3]

\medskip
= map sqr [1,2,3]

\medskip
= map sqr (1:2:3:[])

\medskip
\onslide<2->
= sqr 1 : map sqr (2:3:[])

\medskip
\onslide<3->
= sqr 1 : sqr 2: map sqr (3:[])

\medskip
= sqr 1 : sqr 2: sqr 3: map sqr []

\medskip
= sqr 1 : sqr 2: sqr 3: []

\medskip
= [ 1, 4, 9 ]

\vfill\
\end{frame}

\begin{frame}[fragile]
{Exemplu --- Coduri ASCII}
\begin{block}{Soluție \structure{descriptivă}}
\begin{asciihs}
ords :: [Char] -> [Int]
ords xs = [ ord x | x <- xs ]
\end{asciihs}
\end{block}
\begin{block}{Soluție \structure{recursivă}}
\begin{asciihs}
ords :: [Char] -> [Int]
ords []     = []
ords (x:xs) = ord x : ords xs
\end{asciihs}
\end{block}
\begin{block}{Soluție folosind \lstinline$map$}
\begin{asciihs}
ords :: [Char] -> [Int]
ords xs = map ord xs
\end{asciihs}
\end{block}
\end{frame}

\section{Filter --- Selectarea elementelor dintr-o listă}
\subsection{Exemple}

\begin{frame}[fragile]{Selectarea elementelor pozitive dintr-o listă}
\begin{block}{}
\begin{asciihs}
*Main> positives [1,-2,3]
[1,3]
\end{asciihs}
\end{block}
\begin{block}{Soluție \structure{descriptivă}}
\begin{asciihs}
positives :: [Int] -> [Int]
positives xs = [ x | x <- xs, x > 0 ]
\end{asciihs}
\end{block}
\begin{block}{Soluție \structure{recursivă}}
\begin{asciihs}
positives :: [Int] -> [Int]
positives []                 = []
positives (x:xs) | x > 0     = x : positives xs
                 | otherwise = positives xs
\end{asciihs}
\end{block}
\end{frame}

\begin{frame}[fragile]{Selectarea cifrelor dintr-un șir de caractere}
\begin{block}{}
\begin{asciihs}
*Main> digits "a2c3"
"23"
\end{asciihs}
\end{block}
\begin{block}{Soluție \structure{descriptivă}}
\begin{asciihs}
digits :: [Char] -> [Char]
digits xs = [ x | x <- xs, isDigit x ]
\end{asciihs}
\end{block}
\begin{block}{Soluție \structure{recursivă}}
\begin{asciihs}
digits :: [Char] -> [Char]
digits []                 = []
digits (x:xs) | isDigit x = x : digits xs
              | otherwise = digits xs
\end{asciihs}
\end{block}
\end{frame}

\subsection{Funcția map}

\begin{frame}[fragile]{Funcția \lstinline$filter$}{Definiție}
\begin{block}{}
Date fiind un predicat (funcție booleană) și o listă, selectați elementele din listă care satisfac predicatul.
\end{block}
\begin{block}{Soluție \structure{descriptivă}}
\begin{asciihs}
filter :: (a -> Bool) -> [a] -> [a]
filter p xs = [ x | x <- xs, p x ]
\end{asciihs}
\end{block}
\begin{block}{Soluție \structure{recursivă}}
\begin{asciihs}
filter :: (a -> Bool) -> [a] -> [a]
filter p []                 = []
filter p (x:xs) | p x       = x : filter p xs
                | otherwise = filter p xs
\end{asciihs}
\end{block}
\end{frame}


\begin{frame}[fragile]{Exemplu --- Pozitive}
\begin{block}{Soluție \structure{descriptivă}}
\vspace{-1ex}
\begin{asciihs}
positives :: [Int] -> [Int]
positives xs = [ x | x <- xs, x > 0 ]
\end{asciihs}
\end{block}
\begin{block}{Soluție \structure{recursivă}}
\vspace{-1ex}
\begin{asciihs}
positives :: [Int] -> [Int]
positives []                 = []
positives (x:xs) | x > 0     = x : positives xs
                 | otherwise = positives xs
\end{asciihs}
\end{block}
\begin{block}{Soluție folosind \lstinline$filter$}
\vspace{-1ex}
\begin{asciihs}
positives :: [Int] -> [Int]
positives xs = filter pos xs
  where pos x = x > 0
\end{asciihs}
\end{block}
\end{frame}


\begin{frame}[fragile]{Exemplu --- Cifre}
\begin{block}{Soluție \structure{descriptivă}}
\begin{asciihs}
digits :: [Char] -> [Char]
digits xs = [ x | x <- xs, isDigit x ]
\end{asciihs}
\end{block}
\begin{block}{Soluție \structure{recursivă}}
\begin{asciihs}
digits :: [Char] -> [Char]
digits []                 = []
digits (x:xs) | isDigit x = x : digits xs
              | otherwise = digits xs
\end{asciihs}
\end{block}
\begin{block}{Soluție folosind \lstinline$filter$}
\begin{asciihs}
digits :: [Char] -> [Char]
digits xs = filter isDigit xs
\end{asciihs}
\end{block}
\end{frame}

\section{Fold --- Agregarea elementelor dintr-o listă}

\subsection{Exemple}
\begin{frame}[fragile]{Suma}
\begin{block}{}
Definiți o funcție care dată fiind o listă de numere întregi calculează suma elementelor din listă.
\begin{asciihs}
*Main> sum [1,2,3,4]
10
\end{asciihs}
\end{block}
\begin{block}{Soluție \structure{recursivă}}
\begin{asciihs}
sum :: [Int] -> Int
sum []     = 0
sum (x:xs) = x + sum xs
\end{asciihs}
\end{block}
\end{frame}


\begin{frame}[fragile]{Produs}
\begin{block}{}
Definiți o funcție care dată fiind o listă de numere întregi calculează produsul elementelor din listă.
\begin{asciihs}
*Main> product [1,2,3,4]
24
\end{asciihs}
\end{block}
\begin{block}{Soluție \structure{recursivă}}
\begin{asciihs}
product :: [Int] -> Int
product []     = 1
product (x:xs) = x * sum xs
\end{asciihs}
\end{block}
\end{frame}

\begin{frame}[fragile]{Concatenare}
\begin{block}{}
Definiți o funcție care concatenează o listă de liste.
\begin{asciihs}
*Main> concat [[1,2,3],[4,5]]
[1,2,3,4,5]

*Main> concat ["con","ca","te","na","re"]
"concatenare"
\end{asciihs}
\end{block}
\begin{block}{Soluție \structure{recursivă}}
\begin{asciihs}
concat :: [[a]] -> [a]
concat []       = []
concat (xs:xss) = xs ++ concat xss
\end{asciihs}
\end{block}
\end{frame}

\subsection{Funcția foldr}


\begin{frame}[fragile]{Funcția \lstinline$foldr$}{Definiție}
\begin{block}{}
Date fiind o funcție de actualizare a valorii calculate cu un element curent, o valoare inițială, și o listă, calculați valoare obținută prin aplicarea repetată a funcției de actualizare fiecărui element din listă.
\end{block}
\begin{block}{Soluție \structure{recursivă}}
\begin{asciihs}
foldr :: (a -> b -> b) -> b -> [a] -> b
foldr f i []     = i
foldr f i (x:xs) = f x (foldr i xs)
\end{asciihs}
\end{block}

\onslide<2>
\begin{block}{Soluție \structure{recursivă} cu operator infix}
\begin{asciihs}
foldr :: (a -> b -> b) -> b -> [a] -> b
foldr op i []     = i
foldr op i (x:xs) = x `op` (foldr i xs)
\end{asciihs}
\end{block}
\end{frame}


\begin{frame}[fragile]{Suma}
\begin{block}{Soluție \structure{recursivă}}
\begin{asciihs}
sum :: [Int] -> Int
sum []     = 0
sum (x:xs) = x + sum xs
\end{asciihs}
\end{block}
\begin{block}{Soluție folosind \lstinline$foldr$}
\begin{asciihs}
sum :: [Int] -> Int
sum xs = foldr (+) 0 xs
\end{asciihs}
\end{block}
\end{frame}

\begin{frame}[fragile]{Produs}
\begin{block}{Soluție \structure{recursivă}}
\begin{asciihs}
product :: [Int] -> Int
product []     = 1
product (x:xs) = x * sum xs
\end{asciihs}
\end{block}
\begin{block}{Soluție folosind \lstinline$foldr$}
\begin{asciihs}
product :: [Int] -> Int
product xs = foldr (*) 1 xs
\end{asciihs}
\end{block}
\end{frame}

\begin{frame}[fragile]{Concatenare}
\begin{block}{Soluție \structure{recursivă}}
\begin{asciihs}
concat :: [[a]] -> [a]
concat []       = []
concat (xs:xss) = xs ++ concat xss
\end{asciihs}
\end{block}
\begin{block}{Soluție folosind \lstinline$foldr$}
\begin{asciihs}
concat :: [Int] -> Int
concat xs = foldr (++) [] xs
\end{asciihs}
\end{block}
\end{frame}


\begin{frame}[fragile]{Map în acțiune}{Varianta recursivă}
\begin{asciihs}
sum :: [Int] -> Int
sum xs = foldr (+) 0 xs

foldr :: (a -> b -> b) -> b -> [a] -> b
foldr _ i []      = i
foldr op i (x:xs) = x `op` (foldr i xs)
\end{asciihs}

\vfill
sum [1,2]

\medskip
= foldr (+) 0 [1,2]

\medskip
= foldr (+) 0 (1:2:[])

\medskip
\onslide<2->
= 1 + foldr (+) 0 (2:[])

\medskip
\onslide<3->
= 1 + 2 + 0

\medskip
= 3

\medskip

\vfill\
\end{frame}

\section{Map, Filter, Fold --- combinate}

\begin{frame}[fragile]{Suma pătratelor numerelor pozitive}
\begin{asciihs}
f :: [Int] -> Int
f xs = sum (squares (positives xs))

f :: [Int] -> Int
f xs = sum [ x*x | x <- xs, x > 0 ]

f :: [Int] -> Int
f []                = []
f (x:xs)| x > 0     = (x*x) + f xs
        | otherwise = f xs
        
f :: [Int] -> Int
f xs = foldr (+) 0 (map sqr (filter pos xs))
  where
    sqr x = x * x
    pos x = x > 0
\end{asciihs}
\end{frame}

\section{Currying}

\begin{frame}[fragile]{Adunarea numerelor}
\begin{asciihs}
add :: Int -> Int -> Int
add x y = x + y


  add 3 4
=
  3 + 4
=
  7
\end{asciihs}

\end{frame}

\begin{frame}[fragile]{Adunarea numerelor}
\begin{asciihs}
add :: Int -> (Int -> Int)
(add x) y = x + y


  (add 3) 4
=
  3 + 4
=
  7
\end{asciihs}

\begin{block}{Currying}
A funcție cu două argumente este de fapt o funcție de primul argument
care întoarce o funcție de al doilea argument.
\end{block}
\end{frame}

\begin{frame}[fragile]{Currying}
A funcție cu două argumente este de fapt o funcție de primul argument
care întoarce o funcție de al doilea argument.
\begin{asciihs}
add :: Int -> (Int -> Int)
add x = g
  where
  g y = x + y

  (add 3) 4
=
  g 4
    where
    g y = 3 + y
=
  3 + 4
=
  7
\end{asciihs}
\end{frame}



\begin{frame}[fragile]{Currying}{Haskell Curry (1900--1982)}
\begin{asciihs}
add :: Int -> (Int -> Int)
add x y = x + y
\end{asciihs}
este echivalent (semantic) cu 
\begin{asciihs}
add :: Int -> (Int -> Int)
add x = g
  where
  g y = x + y
\end{asciihs}

\vfill
De asemeni,
\begin{asciihs}
  add 3 4
\end{asciihs}
este echivalent (semantic) cu 
\begin{asciihs}
  (add 3) 4
\end{asciihs}
\end{frame}


\begin{frame}[fragile]{Aplicații Currying --- Stilul funcțional}
\begin{asciihs}
foldr :: (a -> a -> a) -> a -> [a] -> a
foldr f a []     = a
foldr f a (x:xs) = f x (foldr f a xs)

sum :: [Int] -> Int
sum xs = foldr (+) 0 xs
\end{asciihs}
este echivalent (semantic) cu 
\begin{asciihs}
foldr :: (a -> a -> a) -> a -> (*@\color{blue}(@*)[a] -> a(*@\color{blue})@*)
foldr f a []     = a
foldr f a (x:xs) = f x (foldr f a xs)
sum :: [Int] -> Int
sum = foldr (+) 0
\end{asciihs}
\end{frame}


\begin{frame}[fragile]{Aplicații Currying --- Stilul funcțional}
{Suma, Produs, Concatenare}
\begin{asciihs}
sum :: [Int] -> Int
sum = foldr (+) 0

product :: [Int] -> Int
product = foldr (*) 1

concat :: [[a]] -> [a]
concat = foldr (++) []
\end{asciihs}
\end{frame}

\section{Funcții anonime}

\begin{frame}[fragile]{Simplificăm definiția}
\begin{asciihs}
f :: [Int] -> Int
f xs = foldr (+) 0 (map sqr (filter pos xs))
  where
    sqr x = x * x
    pos x = x > 0
\end{asciihs}
\begin{alertblock}{Simplificare incorectă}
\begin{asciihs}
f :: [Int] -> Int
f xs = foldr (+) 0 (map (*@\color{red}{(x * x)} @*) (filter (*@\color{red}{(x > 0)}@*) xs))
\end{asciihs}
\end{alertblock}
\onslide<2>
\begin{block}{Simplificare corectă}
\begin{asciihs}
f :: [Int] -> Int
f xs = foldr (+) 0 
         (map (*@\color{blue}(\textbackslash\; x -> x * x) @*) 
            (filter (*@\color{blue}{(\textbackslash\; x -> x > 0)} @*) xs))
\end{asciihs}
\end{block}
\end{frame}

\begin{frame}[fragile]{Funcții anonime / Lambda Calcul}
\begin{asciihs}
f :: [Int] -> Int
f xs = foldr (+) 0 
         (map (*@\color{blue}(\textbackslash\; x -> x * x) @*) 
            (filter (*@\color{blue}{(\textbackslash\; x -> x > 0)} @*) xs))
\end{asciihs}

\begin{block}{Lambda Calcul}
\begin{itemize}
\item Introdus de logicianul Alonzo Church (1903--1995) pentru dezvoltarea unei teorii a calculabilității
\item În Haskell, \structure{\textbackslash} e folosit în locul simbolului \structure{$\lambda$}
\item Matematic scriem 
\begin{tabular}{c@{ în loc de }c}
$\lambda x. x * x$ & \textbackslash\; x -> x * x
\\
$\lambda x. x > 0$ & \textbackslash\; x -> x > 0
\end{tabular}
\end{itemize}
\end{block}
\end{frame}

\begin{frame}[fragile]{Evaluarea $\lambda$-expresiilor}
\begin{asciihs}
  (\x -> x > 0) 3
=
  let x = 3 in x > 0
=
  3 > 0
=
  True


  (\x -> x * x) 3
=
  let x = 3 in x * x
=
  3 * 3
=
  9
\end{asciihs}
\end{frame}

\begin{frame}[fragile]{Explicație pentru Currying folosind $\lambda$-expresii }
\begin{asciihs}
  (\x -> \y -> x + y) 3 4
=
  ((\x -> (\y -> x + y)) 3) 4
=
  (let x = 3 in \y -> x + y) 4
=
  (\y -> 3 + y) 4
=
  let y = 4 in 3 + y
=
  3 + 4
=
  7
\end{asciihs}
\end{frame}


\begin{frame}[fragile]{Evaluarea $\lambda$-expresiilor}{$\beta$-reducție}
Formula generală pentru evaluarea aplicării $\lambda$-expresiilor este prin substitutirea
argumentului formal cu argumentul actual în corpul funcției:
\[(\lambda x . N)\; M  \xrightarrow{\beta} M[N / x]\]

$\beta$-reducția poate fi descrisă de următoarea identitate Haskell:
\begin{asciihs}
(\ x . n) m == let x = m in n
\end{asciihs}
\end{frame}


\section{Secțiuni (Tăieturi)}

\begin{frame}[fragile]{Secțiuni}
\begin{itemize}
\item \structure{\lstinline$(> 0)$} e forma scurtă a lui \lstinline$(\x -> x > 0)$
\item \structure{\lstinline$(2 *)$} e forma scurtă a lui \lstinline$(\x -> 2 * x)$
\item \structure{\lstinline$(+ 1)$} e forma scurtă a lui \lstinline$(\x -> x + 1)$
\item \structure{\lstinline$(2 ^)$} e forma scurtă a lui \lstinline$(\x -> 2 ^ x)$
\item \structure{\lstinline$(^ 2)$} e forma scurtă a lui \lstinline$(\x -> x ^ 2)$
\item<2> \structure{\lstinline$(`op` 2)$} e forma scurtă a lui \lstinline$(\x -> x `op` 2)$
\item<2> \structure{\lstinline$(2 `op`)$} e forma scurtă a lui \lstinline$(\x -> 2 `op` x)$
\end{itemize}
\end{frame}


\begin{frame}[fragile]{Secțiuni --- Exemplu}
\begin{block}{Folosind $\lambda$-expresii}
\begin{asciihs}
f :: [Int] -> Int
f xs = foldr (+) 0 
         (map (*@\color{blue}(\textbackslash\; x -> x * x) @*) 
            (filter (*@\color{blue}{(\textbackslash\; x -> x > 0)} @*) xs))
\end{asciihs}
\end{block}

\begin{block}{Folosind secțiuni}
\begin{asciihs}
f :: [Int] -> Int
f xs = foldr (+) 0 (map (*@\color{blue}( \^{} 2) @*) (filter (*@\color{blue}{( > 0)} @*) xs))
\end{asciihs}
\end{block}
\end{frame}

\section{Compunerea funcțiilor}


\begin{frame}[fragile]{Compunerea funcțiilor --- operatorul $.$}
\begin{block}{Matematic}
Date fiind $f : A \rightarrow B$ și $g : B \rightarrow C$, compunerea lor, notată
$g \circ f : A \rightarrow C$ este dată de formula \[(g \circ f)(x) = g(f(x))\]
\end{block}

\begin{block}{În Haskell}
\begin{asciihs}
(.) :: (b -> c) -> (a -> b) -> (a -> c)
(g . f) x = g (f x)
\end{asciihs}
\end{block}
\end{frame}

\begin{frame}[fragile]{Operatorul $.$ --- stilul funcțional}
\begin{block}{Definiție cu parametru explicit}
\begin{asciihs}
f :: [Int] -> Int
f xs = foldr (+) 0 (*@\color{blue}(@*)map ( ^ 2)  (*@\color{blue}(@*)filter ( > 0) (*@\color{blue}xs)) @*)
\end{asciihs}
\end{block}

\begin{block}{Definiție compozițională}
\begin{asciihs}
f :: [Int] -> Int
f = foldr (+) 0 . map ( ^ 2) . filter ( > 0)
\end{asciihs}
\end{block}
\end{frame}




\section{Map/Filter/Fold în alte limbaje}

\begin{frame}[fragile]{Map/Filter/Reduce în Javascript}
\begin{lstlisting}[language=JavaScript,basicstyle=\small]
var strs = ["cezara", "petru", "claudia", "virgil"];
var maxLength = strs.filter(function(s){
  return s[0]=='c';
}).map(function(s){
  return s.length;
}).reduce(function(a,b){
  return a>b?a:b;
});
\end{lstlisting}
\end{frame}


\begin{frame}[fragile]{Map/Filter/Reduce în PHP}
\begin{lstlisting}[language=PHP,basicstyle=\small,morekeywords={array_reduce}]
$strs = array("cezara", "petru", "claudia", "virgil");
$max_length = array_reduce(
  array_map(
    "strlen",
    array_filter(
      $strs,
      function($s){return $s[0]=='c';})),
  function($a,$b){
    return $a>$b?$a:$b;
  }, 
  0);
\end{lstlisting}
\end{frame}


\begin{frame}[fragile]{Map/Filter/Reduce în Java 8}
{\url{http://winterbe.com/posts/2014/07/31/java8-stream-tutorial-examples/}}
\begin{lstlisting}[language=Java,basicstyle=\small]
package edu.unibuc.fmi;
import java.util.*;

public class Main {
  public static void main(String[] args) {
    List<String> myList = Arrays.asList(
        "cezara", "petru", "claudia", "virgil");
    int l =
        myList
            .stream()
            .filter(s -> s.startsWith("c"))
            .map(String::length)
            .reduce(0,
                    (l1, l2) -> l1 > l2 ? l1 : l2);
        System.out.println(l);
    }
}
\end{lstlisting}
\end{frame}



\subsection{C++}

\begin{frame}[fragile]{Map/Filter/Reduce în C++11}
{\url{https://meetingcpp.com/tl_files/mcpp/slides/12/FunctionalProgrammingInC++11.pdf}}
\begin{lstlisting}[language=C++,basicstyle=\small]
#include <algorithm>
#include <string>
#include <iostream>
using namespace std;
int main() {
  vector<string>strs {"cezara", "petru", "claudia", "virgil"};
  strs.erase(remove_if(strs.begin(), strs.end(),
                 [](string x){return x[0]!='c';}),
        strs.end());
  vector<int>lengths;   
  transform(strs.begin(), strs.end(), back_inserter(lengths), 
            [](string x) { return x.length();});
  int max_length = accumulate(lengths.begin(), lengths.end(),
                     0,  [](int a,int b){ return a>b?a:b; });     
  cout << max_length;
}
\end{lstlisting}
\end{frame}

\end{document}



