\documentclass[handout,xcolor=pdftex,romanian,colorlinks]{beamer}

\usepackage[export]{adjustbox}
\usepackage{../tslides}
\usepackage[all]{xy}
\usepackage{pgfplots}
\usepackage{flowchart}
\usetikzlibrary{arrows,positioning,calc}
\lstset{language=Haskell}
\lstset{escapeinside={(*@}{@*)}}

\AtBeginSection[]{
  \begin{frame}
  \vfill
  \centering
  \begin{beamercolorbox}[sep=8pt,center,shadow=true,rounded=true]{title}
    \usebeamerfont{title}\insertsectionhead\par%
  \end{beamercolorbox}
  \vfill
  \end{frame}
}


\title[PD---Module si Tipuri]{Programare declarativă\thanks{bazat pe cursul \emph{Informatics 1: Functional Programming} de la \emph{University of Edinburgh}}}
\subtitle{Module si Clase de Tipuri}

\begin{document}
\begin{frame}
  \titlepage
\end{frame}

\section{Module (simple)}

\begin{frame}[fragile]{Module}{Fișierul Geometry.hs}
\begin{asciihs}
module Geometry 
( sphereVolume, sphereArea  
, cuboidVolume, cuboidArea
) where  
      
sphereVolume :: Float -> Float  
sphereVolume radius = (4.0 / 3.0) * pi * (radius ^ 3)  
sphereArea :: Float -> Float  
sphereArea radius = 4 * pi * (radius ^ 2)  

cuboidVolume :: Float -> Float -> Float -> Float  
cuboidVolume a b c = rectangleArea a b * c  
cuboidArea :: Float -> Float -> Float -> Float  
cuboidArea a b c = rectangleArea a b * 2 +
    rectangleArea a c * 2 + rectangleArea c b * 2  
rectangleArea :: Float -> Float -> Float  
rectangleArea a b = a * b  
\end{asciihs}
\end{frame}


\begin{frame}[fragile]{Module Structurate}
\begin{block}
{Fișierul Geometry/Sphere.hs}
\vspace{-2ex}
\begin{asciihs}
module Geometry.Sphere ( volume, area ) where  
volume :: Float -> Float  
volume radius = (4.0 / 3.0) * pi * (radius ^ 3)  
area :: Float -> Float  
area radius = 4 * pi * (radius ^ 2)  
\end{asciihs}
\end{block}
\begin{block}
{Fișierul Geometry/Cuboid.hs}
\vspace{-2ex}
\begin{asciihs}
module Geometry.Cuboid ( volume, area ) where  
volume :: Float -> Float -> Float -> Float  
volume a b c = rectangleArea a b * c  
area :: Float -> Float -> Float -> Float  
area a b c = rectangleArea a b * 2 +
    rectangleArea a c * 2 + rectangleArea c b * 2  
rectangleArea :: Float -> Float -> Float  
rectangleArea a b = a * b  
\end{asciihs}
\end{block}
\end{frame}

\begin{frame}[fragile]{Importare: redenumire și ascundere}
\begin{block}
{Fișierul Geometry/Cube.hs}
\vspace{-2ex}
\begin{asciihs}
module Geometry.Cube ( volume, area ) where  

import qualified Geometry.Cuboid as Cuboid

volume :: Float -> Float  
volume side = Cuboid.volume side side side  
  
area :: Float -> Float  
area side = Cuboid.area side side side  
\end{asciihs}
\end{block}
\begin{block}
{Fișierul Main.hs}
\vspace{-2ex}
\begin{asciihs}
import qualified Geometry.Sphere as Sphere (volume)
import Geometry.Cuboid (volume)
import Geometry.Cube hiding (volume)
\end{asciihs}
\end{block}
\end{frame}

\section{Abstractizare}

\subsection{Mulțimi}

\begin{frame}[fragile]{Mulțimi implementate folosind liste}
{Fișierul SetListImpl.hs\hfill (1)}
\begin{asciihs}
module SetListImpl
(Set,empty,insert,set,element,equal,check) where
import Test.QuickCheck

type Set a = [a]

empty :: Set a
empty = []

insert :: a -> Set a -> Set a
insert x xs = x:xs

set :: [a] -> Set a
set xs = xs
\end{asciihs}
\end{frame}


\begin{frame}[fragile]{Mulțimi implementate folosind liste}
{Fișierul SetListImpl.hs\hfill (2)}
\begin{asciihs}
element :: Eq a => a -> Set a -> Bool
x `element` xs = x `elem` xs

equal :: Eq a => Set a -> Set a -> Bool
xs `equal` ys = xs `subset` ys && ys `subset` xs
  where
    xs `subset` ys = and [ x `elem` ys | x <- xs ]
\end{asciihs}
\end{frame}

\begin{frame}[fragile]{Mulțimi implementate folosind liste}
{Proprietăți și teste}
\begin{block}{Fișierul SetListImpl.hs\hfill (3)}
\begin{asciihs}
prop_element :: [Int] -> Bool
prop_element ys =
    and [ x `element` s == odd x | x <- ys ]
  where
    s = set [ x | x <- ys, odd x ]
check =
    quickCheck prop_element
\end{asciihs}
\end{block}
\begin{block}{GHCI}
\begin{asciihs}
SetListImpl> check
+++ OK, passed 100 tests.
\end{asciihs}
\end{block}
\end{frame}

\begin{frame}[fragile]{Probleme de abstractizare}
{Fișierul SetListImplTest.hs}
\begin{asciihs}
module SetListImplTest where
import SetListImpl
test :: Int -> Bool
test n =
   s `equal` t
  where
    s = set [1,2..n]
    t = set [n,n-1..1]

breakAbstraction :: Set a -> a
\end{asciihs}
\onslide<2->
\vspace{-2ex}
\begin{asciihs}
breakAbstraction = head
\end{asciihs}
\onslide<3->
\begin{itemize}
\item Nu e funcție
\end{itemize}
\onslide<4>
\begin{asciihs}
head (set [1,2,3]) == 1 /= 3 == head (set [3,2,1])
\end{asciihs}
\end{frame}


%
%ListUnabs.hs (3)
%
%
%
%Part III
%
%Sets as ordered lists
%without abstraction
%
%OrderedListUnabs.hs (1)
%module OrderedListUnabs
%(Set,empty,insert,set,element,equal,check) where
%import Data.List(nub,sort)
%import Test.QuickCheck
%type Set a = [a]
%invariant :: Ord a => Set a -> Bool
%invariant xs =
%and [ x < y | (x,y) <- zip xs (tail xs) ]
%
%OrderedListUnabs.hs (2)
%empty :: Set a
%empty = []
%insert :: Ord a => a -> Set a -> Set a
%insert x []
%= [x]
%insert x (y:ys) | x < y
%= x : y : ys
%| x == y = y : ys
%| x > y
%= y : insert x ys
%set :: Ord a => [a] -> Set a
%set xs = nub (sort xs)
%
%OrderedListUnabs.hs (3)
%element :: Ord a => a -> Set a -> Bool
%x ‘element‘ []
%= False
%x ‘element‘ (y:ys) | x < y
%= False
%| x == y
%= True
%| x > y
%= x ‘element‘ ys
%equal :: Eq a => Set a -> Set a -> Bool
%xs ‘equal‘ ys = xs == ys
%
%OrderedListUnabs.hs (4)
%prop_invariant :: [Int] -> Bool
%prop_invariant xs = invariant s
%where
%s = set xs
%prop_element :: [Int] -> Bool
%prop_element ys =
%and [ x ‘element‘ s == odd x | x <- ys ]
%where
%s = set [ x | x <- ys, odd x ]
%check =
%quickCheck prop_invariant >>
%quickCheck prop_element
%Prelude OrderedListUnabs> check
%+++ OK, passed 100 tests.
%+++ OK, passed 100 tests.
%
%OrderedListUnabsTest.hs
%module OrderedListUnabsTest where
%import OrderedListUnabs
%test :: Int -> Bool
%test n =
%s ‘equal‘ t
%where
%s = set [1,2..n]
%t = set [n,n-1..1]
%breakAbstraction :: Set a -> a
%breakAbstraction = head
%-- now it’s a function
%-- head (set [1,2,3]) == 1 == head (set [3,2,1])
%badtest :: Int -> Bool
%badtest n =
%s ‘equal‘ t
%where
%s = [1,2..n]
%-- no call to set!
%t = [n,n-1..1]
%-- no call to set!
%
%Part IV
%
%Sets as ordered trees
%without abstraction
%
%TreeUnabs.hs (1)
%module TreeUnabs
%(Set(Nil,Node),empty,insert,set,element,equal,check) where
%import Test.QuickCheck
%data
%
%Set a
%
%=
%
%Nil | Node (Set a) a (Set a)
%
%list :: Set a -> [a]
%list Nil
%=
%list (Node l x r) =
%
%[]
%list l ++ [x] ++ list r
%
%invariant :: Ord a => Set a -> Bool
%invariant Nil
%= True
%invariant (Node l x r) =
%invariant l && invariant r &&
%and [ y < x | y <- list l ] &&
%and [ y > x | y <- list r ]
%
%TreeUnabs.hs (2)
%empty :: Set a
%empty = Nil
%insert :: Ord a => a -> Set a -> Set a
%insert x Nil = Node Nil x Nil
%insert x (Node l y r)
%| x == y
%= Node l y r
%| x < y
%= Node (insert x l) y r
%| x > y
%= Node l y (insert x r)
%set :: Ord a => [a] -> Set a
%set = foldr insert empty
%
%TreeUnabs.hs (3)
%element :: Ord a => a -> Set a -> Bool
%x ‘element‘ Nil = False
%x ‘element‘ (Node l y r)
%| x == y
%= True
%| x < y
%= x ‘element‘ l
%| x > y
%= x ‘element‘ r
%equal :: Ord a => Set a -> Set a -> Bool
%s ‘equal‘ t = list s == list t
%
%TreeUnabs.hs (4)
%prop_invariant :: [Int] -> Bool
%prop_invariant xs = invariant s
%where
%s = set xs
%prop_element :: [Int] -> Bool
%prop_element ys =
%and [ x ‘element‘ s == odd x | x <- ys ]
%where
%s = set [ x | x <- ys, odd x ]
%check =
%quickCheck prop_invariant >>
%quickCheck prop_element
%-- Prelude TreeUnabs> check
%-- +++ OK, passed 100 tests.
%-- +++ OK, passed 100 tests.
%
%TreeUnabsTest.hs
%module TreeUnabsTest where
%import TreeUnabs
%test :: Int -> Bool
%test n =
%s ‘equal‘ t
%where
%s = set [1,2..n]
%t = set [n,n-1..1]
%badtest :: Bool
%badtest =
%s ‘equal‘ t
%where
%s = set [1,2,3]
%t = Node (Node Nil 3 Nil) 2 (Node Nil 1 Nil)
%-- breaks the invariant!
%
%Part V
%
%Sets as balanced trees
%without abstraction
%
%BalancedTreeUnabs.hs (1)
%module BalancedTreeUnabs
%(Set(Nil,Node),empty,insert,set,element,equal,check) where
%import Test.QuickCheck
%type
%data
%
%Depth
%Set a
%
%=
%=
%
%Int
%Nil | Node (Set a) a (Set a) Depth
%
%node :: Set a -> a -> Set a -> Set a
%node l x r = Node l x r (1 + (depth l ‘max‘ depth r))
%depth :: Set a -> Int
%depth Nil = 0
%depth (Node _ _ _ d) = d
%
%BalancedTreeUnabs.hs (2)
%list :: Set a -> [a]
%list Nil
%=
%list (Node l x r _) =
%
%[]
%list l ++ [x] ++ list r
%
%invariant :: Ord a => Set a -> Bool
%invariant Nil
%= True
%invariant (Node l x r d) =
%invariant l && invariant r &&
%and [ y < x | y <- list l ] &&
%and [ y > x | y <- list r ] &&
%abs (depth l - depth r) <= 1 &&
%d == 1 + (depth l ‘max‘ depth r)
%
%BalancedTreeUnabs.hs (3)
%empty :: Set a
%empty = Nil
%insert :: Ord a => a -> Set a -> Set a
%insert x Nil = node empty x empty
%insert x (Node l y r _)
%| x == y
%= node l y r
%| x < y
%= rebalance (node (insert x l) y r)
%| x > y
%= rebalance (node l y (insert x r))
%set :: Ord a => [a] -> Set a
%set = foldr insert empty
%
%Rebalancing
%y
%
%x
%
%x
%
%y
%C
%A
%
%B
%
%B
%
%C
%
%A
%
%Node (Node a x b) y c
%
%-->
%
%Node a x (Node b y c)
%
%y
%
%z
%x
%
%x
%
%z
%
%y
%D
%A
%
%B
%A
%
%B
%
%C
%
%D
%
%C
%
%Node (Node a x (Node b y c) z d)
%--> Node (Node a x b) y (Node c z d)
%
%BalancedTreeUnabs.hs (4)
%rebalance :: Set a -> Set a
%rebalance (Node (Node a x b _) y c _)
%| depth a >= depth b && depth a > depth c
%= node a x (node b y c)
%rebalance (Node a x (Node b y c _) _)
%| depth c >= depth b && depth c > depth a
%= node (node a x b) y c
%rebalance (Node (Node a x (Node b y c _) _) z d _)
%| depth (node b y c) > depth d
%= node (node a x b) y (node c z d)
%rebalance (Node a x (Node (Node b y c _) z d _) _)
%| depth (node b y c) > depth a
%= node (node a x b) y (node c z d)
%rebalance a = a
%
%BalancedTreeUnabs.hs (5)
%element :: Ord a => a -> Set a -> Bool
%x ‘element‘ Nil = False
%x ‘element‘ (Node l y r _)
%| x == y
%= True
%| x < y
%= x ‘element‘ l
%| x > y
%= x ‘element‘ r
%equal :: Ord a => Set a -> Set a -> Bool
%s ‘equal‘ t = list s == list t
%
%BalancedTreeUnabs.hs (6)
%prop_invariant :: [Int] -> Bool
%prop_invariant xs = invariant s
%where
%s = set xs
%prop_element :: [Int] -> Bool
%prop_element ys =
%and [ x ‘element‘ s == odd x | x <- ys ]
%where
%s = set [ x | x <- ys, odd x ]
%check =
%quickCheck prop_invariant >>
%quickCheck prop_element
%-- Prelude BalancedTreeUnabs> check
%-- +++ OK, passed 100 tests.
%-- +++ OK, passed 100 tests.
%
%BalancedTreeUnabsTest.hs
%module BalancedTreeUnabsTest where
%import BalancedTreeUnabs
%test :: Int -> Bool
%test n =
%s ‘equal‘ t
%where
%s = set [1,2..n]
%t = set [n,n-1..1]
%badtest :: Bool
%badtest =
%s ‘equal‘ t
%where
%s = set [1,2,3]
%t = (Node Nil 1 (Node Nil 2 (Node Nil 3 Nil 1) 2) 3)
%-- breaks the invariant!
%
%Part VI
%
%Complexity, revisited
%
%Summary
%insert
%
%set
%
%element
%
%equal
%
%List
%
%O(1)
%
%O(1)
%
%O(n)
%
%O(n2 )
%
%OrderedList
%
%O(n)
%
%O(n log n)
%
%O(n)
%
%O(n)
%
%O(log n)∗
%
%O(n log n)∗
%
%O(log n)∗
%
%Tree
%BalancedTree
%
%†
%
%2 †
%
%†
%
%O(n)
%
%O(n )
%
%O(n)
%
%O(log n)
%
%O(n log n)
%
%O(log n)
%
%* average case
%
%/ † worst case
%
%O(n)
%O(n)
%
%Part VII
%
\section{Încapsulare}

\begin{frame}[fragile]{Încapsulare}
{Fișierul SetListAbs.hs\hfill (1)}
\begin{asciihs}
module SetListAbs
(Set,empty,insert,set,element,equal,check) where
import Test.QuickCheck

data Set a   =  MkSet [a]

empty :: Set a
empty = MkSet []

insert :: a -> Set a -> Set a
insert x (MkSet xs) = MkSet (x:xs)

set :: [a] -> Set a
set xs = MkSet xs
\end{asciihs}
\end{frame}


\begin{frame}[fragile]{Încapsulare}
{Fișierul SetListAbs.hs\hfill (2)}
\begin{asciihs}
element :: Eq a => a -> Set a -> Bool
x `element` (MkSet xs) = x `elem` xs

equal :: Eq a => Set a -> Set a -> Bool
MkSet xs `equal` MkSet ys =
    xs `subset` ys && ys `subset` xs
  where
    xs `subset` ys = and [ x `elem` ys | x <- xs ]
\end{asciihs}
\end{frame}


\begin{frame}[fragile]{Încapsulare}
{Proprietăți și teste}
\begin{block}{Fișierul SetListAbs.hs\hfill (3)}
\begin{asciihs}
prop_element :: [Int] -> Bool
prop_element ys =
    and [ x `element` s == odd x | x <- ys ]
  where
    s = set [ x | x <- ys, odd x ]

check =
    quickCheck prop_element
\end{asciihs}
\end{block}
\begin{block}{GHCI}
\begin{asciihs}
SetListAbs> check
+++ OK, passed 100 tests.
\end{asciihs}
\end{block}
\end{frame}


\begin{frame}[fragile]{Probleme de abstractizare?}
{Fișierul SetListAbsTest.hs}
\begin{asciihs}
module SetListAbsTest where
import SetListAbs
test :: Int -> Bool
test n =
   s `equal` t
  where
    s = set [1,2..n]
    t = set [n,n-1..1]

breakAbstraction :: Set a -> a
\end{asciihs}
\onslide<2->
\vspace{-2ex}
\begin{asciihs}
breakAbstraction = head   --  eroare de tipuri
\end{asciihs}
\end{frame}

\begin{frame}[fragile]{Încapsulare = Ascundere de informație}
\begin{block}{\lstinline"module ListAbs(Set,empty,insert,set,element,equal)"}
\begin{asciihs}
> ghci SetListAbs.hs
Ok, modules loaded: SetListAbs
*SetListAbs> let s0 = set [2,7,1,8,2,8]
*SetListAbs> let MkSet xs = s0 in xs
Not in scope: data constructor `MkSet`
\end{asciihs}
\end{block}
\begin{block}{\lstinline"module SetListAbs(Set(MkSet),empty,insert,set,element,equal)"}
\begin{asciihs}
> ghci SetListAbs.hs
*SetListAbs> let s0 = set [2,7,1,8,2,8]
*SetListAbs> let MkSet xs = s0 in xs 
[2,7,1,8,2,8]
*SetListAbs> head xs
2
\end{asciihs}
\end{block}
\end{frame}
%Hiding—the secret of abstraction
%
%vs.
%module ListUnhidden(Set(MkSet),empty,insert,element,equal)
%
%Hiding—the secret of abstraction
%module TreeAbs(Set,empty,insert,set,element,equal)
%> ghci TreeAbs.hs
%Ok, modules loaded: SetList, MainList.
%*TreeAbs> let s0 = Node (Node Nil 3 Nil) 2 (Node Nil 1 Nil)
%Not in scope: data constructor ‘Node’, ‘Nil’
%
%vs.
%module TreeUnabs(Set(Node,Nil),empty,insert,element,equal)
%> ghci TreeUnabs.hs
%*SetList> let s0 = Node (Node Nil 3 Nil) 2 (Node Nil 1 Nil)
%*SetList> invariant s0
%False
%
%Preserving the invariant
%module TreeAbsInvariantTest where
%import TreeAbs
%prop_invariant_empty = invariant empty
%prop_invariant_insert x s =
%invariant s ==> invariant (insert x s)
%prop_invariant_set xs = invariant (set xs)
%check =
%quickCheck prop_invariant_empty >>
%quickCheck prop_invariant_insert >>
%quickCheck prop_invariant_set
%-----
%
%Prelude
%+++ OK,
%+++ OK,
%+++ OK,
%
%TreeAbsInvariantTest> check
%passed 1 tests.
%passed 100 tests.
%passed 100 tests.

\section{Clase de tipuri}

\begin{frame}[fragile]{Test de apartenență}
\begin{asciihs}
elem :: Eq a => a -> [a] -> Bool
\end{asciihs}

\begin{block}{Folosind descrieri de liste}
\onslide<2>
\begin{asciihs}
elem x ys     = or [ x == y | y <- ys ]
\end{asciihs}
\end{block}

\onslide<1->
\begin{block}{Folosind recursivitate}
\onslide<2>
\begin{asciihs}
elem x []     = False
elem x (y:ys) = x == y || elem x ys
\end{asciihs}
\end{block}

\onslide<1->
\begin{block}{Folosind funcții de nivel înalt}
\onslide<2>
\begin{asciihs}
elem x ys     = foldr (||) False (map (x ==) ys)
\end{asciihs}
\end{block}
\end{frame}

\begin{frame}[fragile]{Funcția elem este polimorfică}{Dar nu pentru orice tip}
\begin{asciihs}
*Main> elem 1 [2,3,4]
False

*Main> elem `o` "word"
True

*Main> elem (1,`o`) [(0,`w`),(1,`o`),(2,`r`),(3,`d`)]
True

*Main> elem "word" ["list","of","word"]
True

*Main> elem (\x -> x) [(\x -> -x), (\x -> -(-x))]
No instance for (Eq (a -> a)) arising from a use of `elem`
Possible fix: add an instance declaration for (Eq (a -> a))
\end{asciihs}
\end{frame}

\begin{frame}[fragile]{Clasa de tipuri pentru egalitate}
\begin{asciihs}
  class Eq a where
    (==) :: a -> a -> Bool

  instance Eq Int      where
    (==) = eqInt

  instance   Eq Char    where
    x == y             = ord x == ord y

  instance (Eq a, Eq b) => Eq (a,b) where
    (u,v) == (x,y)     = (u == x) && (v == y)

  instance Eq a => Eq [a] where
    [] == []           = True
    [] == y:ys         = False
    x:xs == []         = False
    x:xs == y:ys       = (x == y) && (xs == ys)
\end{asciihs}
\end{frame}

\begin{frame}[fragile]{Clasă de tipuri = dicționar de funcții}
\begin{asciihs}
type EqDict a =  EqD { eq :: a -> a -> Bool }
elem :: (*@\color{blue}\tt EqDict a ->@*) a -> [a] -> Bool
\end{asciihs}

\begin{block}{Folosind descrieri de liste}
\onslide<2>
\begin{asciihs}
elem (*@\color{blue}\tt (EqD eq)@*) x ys     = or [ x (*@\color{blue}\tt `eq`@*) y | y <- ys ]
\end{asciihs}
\end{block}

\onslide<1->
\begin{block}{Folosind recursivitate}
\onslide<2>
\begin{asciihs}
elem (*@\color{blue}\tt \_ @*) x []          = False
elem (*@\color{blue}\tt (EqD eq)@*) x (y:ys) = x (*@\color{blue}\tt `eq`@*) y || elem x ys
\end{asciihs}
\end{block}

\onslide<1->
\begin{block}{Folosind funcții de nivel înalt}
\onslide<2>
\begin{asciihs}
elem (*@\color{blue}\tt d@*) x ys     = foldr (||) False (map ((*@\color{blue}\tt eq d@*) x) ys)
\end{asciihs}
\end{block}
\end{frame}

\begin{frame}[fragile]{Clasă de tipuri = dicționar de funcții\hfill Instanțieri}
\begin{asciihs}
dInt :: EqDict Int
dInt = EqD eqInt

dChar ::  EqDict Char
dChar = EqD (\ x y -> ord x == ord y)

dPair :: (EqDict a, EqDict b) -> EqDict (a,b)
dPair (EqD eqa, EqD eqb) = EqD eq
  where (u,v) `eq` (x,y) = (u `eqa` x) && (v `eqb` y)

dList :: EqDict a -> EqDict [a]
dList (EqD eqa) = EqD eq
  where 
    [] `eq` []         = True
    [] `eq` (y:ys)     = False
    (x:xs) `eq []      = False
    (x:xs) `eq` (y:ys) = (x `eqa` y) && (xs `eq` ys)
\end{asciihs}
\end{frame}

\begin{frame}[fragile]{Polimorfism cu dicționare de funcții}
\begin{asciihs}
*Main> elem dInt 1 [2,3,4]
False

*Main> elem dChar `o` "word"
True

*Main> elem (dPair (dInt,dChar)) 
            [(1,`o`) [(0,`w`),(1,`o`),(2,`r`),(3,`d`)]
True

*Main> elem (dList dChar) "word" ["list","of","word"]
True
\end{asciihs}
\end{frame}



\section{Eq, Ord, Show}
\begin{frame}[fragile]{Eq, Ord, Show}
\begin{asciihs}
  class  Eq a  where
    (==) :: a -> a -> Bool
    (/=) :: a -> a -> Bool
    -- minimum definition: (==)
    x /= y = not (x == y)

  class  (Eq a) => Ord a  where
    (<)  ::   a -> a ->   Bool
    (<=) ::   a -> a ->   Bool
    (>)  ::   a -> a ->   Bool
    (>=) ::   a -> a ->   Bool
    -- minimum   definition: (<=)
    x < y   =    x <= y && x /= y
    x > y   =    y < x
    x >= y  =     y <= x

  class  Show a  where
    show :: a -> String
\end{asciihs}
\end{frame}

\subsection{Instanțieri pentru Bool, perechi, liste}
\begin{frame}[fragile]{Bool}
\begin{asciihs}
  instance  Eq Bool  where
    False == False = True
    False == True  = False
    True  == False = False
    True  == True  = True

  instance  Ord Bool  where
    False <= False = True
    False <= True  = True
    True  <= False = False
    True  <= True  = True

  instance  Show Bool  where
    show False      = "False"
    show True       = "True"
\end{asciihs}
\end{frame}
\begin{frame}[fragile]{Perechi}
\begin{asciihs}
instance (Eq a, Eq b) => Eq (a,b) where
  (x,y) == (x',y') = x == x' && y == y'

instance (Ord a, Ord b) => Ord (a,b) where
  (x,y) <= (x',y') = x < x' || (x == x' && y <= y')

instance (Show a, Show b) => Show (a,b) where
  show (x,y) = "(" ++ show x ++ "," ++ show y ++ ")"
\end{asciihs}
\end{frame}

\begin{frame}[fragile]{Liste}
\begin{asciihs}
  instance  Eq a => Eq [a]  where
    []     == []     = True
    []     == (y:ys) = False
    (x:xs) == []     = False
    (x:xs) == (y:ys) = x == y && xs == ys

  instance  Ord a => Ord [a]  where
    []     <= ys     = True
    (x:xs) <= []     = False
    (x:xs) <= (y:ys) = x < y || (x == y && xs <= ys)

  instance Show a => Show [a] where
    show []      = "[]"
    show (x:xs) = "[" ++ showSep x xs ++ "]"
      where
        showSep x []      = show x
        showSep x (y:ys) = show x ++ "," ++ showSep y ys
\end{asciihs}
\end{frame}
\begin{frame}[fragile]{Derivare automata pentru tipuri algebrice}
\begin{asciihs}
  data Bool = False | True
        deriving (Eq, Ord, Show)

  data Pair a b = MkPair a b
        deriving (Eq, Ord, Show)

  data List a = Nil | Cons a (List a)
        deriving (Eq, Ord, Show)
\end{asciihs}
\end{frame}

\begin{frame}[fragile]{Mulțimi ca instanță a lui Eq}
\begin{asciihs}
  instance Eq (Set a) where
    s == t = s `equal` t
\end{asciihs}

\begin{block}{Observație}
\begin{itemize}
\item Diferit față de implementarea implicită dată de \lstinline"deriving"
\item Deoarece egalitatea de mulțimi e mai mult decât egalitatea sintactică
\end{itemize}
\end{block}
\end{frame}

\section{Numere}
\begin{frame}[fragile]{Clase de tipuri pentru numere}
\vspace{-2ex}
\begin{asciihs}
  class (Eq a, Show a) => Num a where
    (+),(-),(*)   :: a -> a -> a
    negate        :: a -> a
    fromInteger   :: Integer -> a
    -- minimum definition: (+),(-),(*),fromInteger
    negate x      =   fromInteger 0 - x
(*@\vspace{-1ex}@*)
  class (Num a) => Fractional a where
    (/)           :: a -> a -> a
    recip         :: a -> a
    fromRational :: Rational -> a
    -- minimum definition: (/), fromRational
    recip x       =   1/x
(*@\vspace{-1ex}@*)
  class (Num a, Ord a) => Real a where
    toRational    :: a -> Rational
(*@\vspace{-1ex}@*)
  class (Real a, Enum a) => Integral a where
    div, mod      :: a -> a -> a
    toInteger     :: a -> Integer
\end{asciihs}
\end{frame}
\begin{frame}[fragile]{Instanțe pentru tipul predefinit Float}
\begin{asciihs}
  instance Num Float   where
    (+)           =    builtInAddFloat
    (-)           =    builtInSubtractFloat
    (*)           =    builtInMultiplyFloat
    negate        =    builtInNegateFloat
    fromInteger   =    builtInFromIntegerFloat

  instance Fractional Float where
    (/)           = builtInDivideFloat
    fromRational = builtInFromRationalFloat
\end{asciihs}
\end{frame}
\begin{frame}[fragile]{Să definim numerele naturale}{Fișierul Natural.hs \hfill(1)}
\begin{asciihs}
  module Natural(Nat) where
  import Test.QuickCheck

  data Nat = MkNat Integer

  invariant :: Nat -> Bool
  invariant (MkNat x) = x >= 0

  instance Eq Nat where
    MkNat x == MkNat y = x == y

  instance Ord Nat where
    MkNat x <= MkNat y = x <= y

  instance Show Nat where
    show (MkNat x) = show x
\end{asciihs}
\end{frame}
\begin{frame}[fragile]{Naturale ca instanță a lui Num}
{Fișierul Natural.hs \hfill(2)}
\begin{asciihs}
instance Num Nat where
  MkNat x + MkNat y  =  MkNat (x + y)
  MkNat x - MkNat y
    | x >= y    =  MkNat (x - y)
    | otherwise =  error (show (x-y) ++ " is negative")
  MkNat x * MkNat y  =  MkNat (x * y)
  fromInteger x
    | x >= 0    =  MkNat x
    | otherwise =  error (show x ++ " is negative")
  negate  =   undefined
\end{asciihs}
\end{frame}
\begin{frame}[fragile]{Teste de consistență}
{Fișierul Natural.hs \hfill(3)}
\vspace{-2ex}
\begin{asciihs}
  prop_plus :: Integer -> Integer -> Property
  prop_plus m n =
    (m >= 0) && (n >= 0) ==> (m+n >= 0)
(*@\vspace{-1ex}@*)
  prop_times :: Integer -> Integer -> Property
  prop_times m n =
    (m >= 0) && (n >= 0) ==> (m*n >= 0)
(*@\vspace{-1ex}@*)
  prop_minus :: Integer -> Integer -> Property
  prop_minus m n =
    (m >= 0) && (n >= 0) && (m >= n) ==> (m-n >= 0)
\end{asciihs}

\begin{block}{Fișierul NaturalTest.hs}
\vspace{-2ex}
\begin{asciihs}
  module NaturalTest where
  import Natural
(*@\vspace{-1ex}@*)
  m, n :: Nat
  m = fromInteger 2
  n = fromInteger 3
\end{asciihs}
\end{block}
\end{frame}
\begin{frame}[fragile]{Testare}
\begin{asciihs}
 > ghci NaturalTest
 Ok, modules loaded: NaturalTest, Natural.
 *NaturalTest> m
 2
 *NaturalTest> n
 3
 *NaturalTest> m+n
 5
 *NaturalTest> n-m
 1
 *NaturalTest> m-n
 *** Exception: -1 is negative
 *NaturalTest> m*n
 6
 *NaturalTest> fromInteger (-5) :: Nat
 *** Exception: -5 is negative
 *NaturalTest> MkNat (-5)
 Not in scope: data constructor `MkNat`
\end{asciihs}
\end{frame}


%\section{Enumerări}
%
%\begin{frame}[fragile]{Clasa de tipuri Enum}
%\vspace{-2ex}
%\begin{asciihs}
%  class Enum a where
%    toEnum         ::   Int -> a
%    fromEnum       ::   a -> Int
%    succ, pred     ::   a -> a
%    enumFrom       ::   a -> [a]             --   [x..]
%    enumFromTo     ::   a -> a -> [a]        --   [x..y]
%    enumFromThen   ::   a -> a -> [a]        --   [x,y..]
%    enumFromThenTo ::   a -> a -> a -> [a]   --   [x,y..z]
%   -- minimum definition: toEnum, fromEnum
%   succ x           = toEnum (fromEnum x + 1)
%   pred x           = toEnum (fromEnum x - 1)
%   enumFrom x
%     = map toEnum [fromEnum x ..]
%   enumFromTo x y
%     = map toEnum [fromEnum x .. fromEnum y]
%   enumFromThen x y
%     = map toEnum [fromEnum x, fromEnum y ..]
%   enumFromThenTo x y z
%     = map toEnum [fromEnum x, fromEnum y .. fromEnum z]
%\end{asciihs}
%\end{frame}
%\begin{frame}[fragile]{Int ca instanță a lui Enum}
%\begin{asciihs}
%  instance Enum Int where
%    toEnum x          = x
%    fromEnum x        = x
%    succ x            = x+1
%    pred x            = x-1
%    enumFrom x        = iterate (+1) x
%    enumFromTo x y    = takeWhile (<= y) (iterate (+1) x)
%    enumFromThen x y  = iterate (+(y-x)) x
%    enumFromThenTo x y z
%        = takeWhile (<= z) (iterate (+(y-x)) x)
%
%  iterate :: (a -> a) -> a -> [a]
%  iterate f x = x : iterate f (f x)
%
%  takeWhile :: (a -> Bool) -> [a]   ->   [a]
%  takeWhile p []                    =    []
%  takeWhile p (x:xs) | p x          =    x : takeWhile p xs
%                     | otherwise    =    []
%\end{asciihs}
%\end{frame}
%
%\subsection{Anotimpuri}
%
%\begin{frame}[fragile]{Anotimpuri}
%\begin{asciihs}
%  data Season = Winter | Spring | Summer | Fall
%
%  next   :: Season -> Season
%  next   Winter = Spring
%  next   Spring = Summer
%  next   Summer = Fall
%  next   Fall    = Winter
%
%  warm   :: Season -> Bool
%  warm   Winter = False
%  warm   Spring = True
%  warm   Summer = True
%  warm   Fall    = True
%\end{asciihs}
%\end{frame}
%
%\begin{frame}[fragile]{Anotimpuri --- Instanțiere manuala pentru Eq, Ord, Show}
%\vspace{-2ex}
%\begin{asciihs}
%  instance   Eq   Seasons   where
%    Winter   ==   Winter    = True
%    Spring   ==   Spring    = True
%    Summer   ==   Summer    = True
%    Fall     ==   Fall      = True
%    _        ==   _         = False
%  instance   Ord Seasons where
%    Spring   <= Winter = False
%    Summer   <= Winter = False
%    Summer   <= Spring = False
%    Fall     <= Winter = False
%    Fall     <= Spring = False
%    Fall     <= Summer = False
%    _        <= _       = True
%  instance Show Seasons where
%    show Winter = "Winter"
%    show Spring = "Spring"
%    show Summer = "Summer"
%    show Fall   = "Fall"
%\end{asciihs}
%\end{frame}
%
%\begin{frame}[fragile]{Instanță Enum pentru anotimpuri}
%\begin{asciihs}
%  instance Enum Seasons where
%
%    fromEnum     Winter   =    0
%    fromEnum     Spring   =    1
%    fromEnum     Summer   =    2
%    fromEnum     Fall     =    3
%
%    toEnum   0    =   Winter
%    toEnum   1    =   Spring
%    toEnum   2    =   Summer
%    toEnum   3    =   Fall
%\end{asciihs}
%\end{frame}
%
%\begin{frame}[fragile]{Anotimpuri folosind derivare automată}
%\begin{asciihs}
%  data Season = Winter | Spring | Summer | Fall
%                deriving (Eq, Ord, Show, Enum)
%
%  next :: Season -> Season
%  next x = toEnum ((fromEnum x + 1) `mod` 4)
%
%  warm :: Season -> Bool
%  warm x = x `elem` [Spring .. Fall]
%\end{asciihs}
%
%\vfill
%Toate funcțiile pentru clasele derivate sunt generate automat!
%\end{frame}
%
%\begin{frame}[fragile]{}
%\begin{asciihs}
%\end{asciihs}
%\end{frame}


%Part VII
%
%Shape
%Shape
%  type   Radius   =   Float
%  type   Width    =   Float
%  type   Height   =   Float
%
%  data   Shape    =   Circle Radius
%                  |   Rect Width Height
%
%  area :: Shape -> Float
%  area (Circle r) = pi * rˆ2
%  area (Rect w h) = w * h
%Eq, Ord, Show
%  instance Eq Shape where
%    Circle r == Circle r’  =   r == r’
%    Rect w h == Rect w’ h’ =   w == w’ && h == h’
%    _        == _          =   False
%
%  instance Ord Shape where
%    Circle r <= Circle r’  =   r < r’
%    Circle r <= Rect w’ h’ =   True
%    Rect w h <= Rect w’ h’ =   w < w’ || (w == w’ && h <= h’)
%    _        <= _          =   False
%
%  instance Show Shape where
%    show (Circle r)   = "Circle " ++ showN r
%    show (Radius w h) = "Radius " ++ showN w ++ " " ++ showN h
%
%  showN :: (Num a) => a -> String
%  showN x | x >= 0     = show x
%          | otherwise = "(" ++ show x ++ ")"
%Deriving Shapes
%  data   Shape   = Circle Radius
%                 | Rect Width Height
%                 deriving (Eq, Ord, Show)
%
%
%
%
%         Haskell uses types to write code for you!
%  Part VIII
%
%Expressions
%Expression Trees
%  data   Exp   = Lit Int
%               | Exp :+: Exp
%               | Exp :*: Exp
%
%  eval   :: Exp -> Int
%  eval   (Lit n)    = n
%  eval   (e :+: f) = eval e + eval f
%  eval   (e :*: f) = eval e * eval f
%
%  *Main> eval (Lit 2 :+: (Lit 3 :*: Lit 3))
%  11
%  *Main> eval ((Lit 2 :+: Lit 3) :*: Lit 3)
%  15
%Eq, Ord, Show
%  instance Eq Exp where
%    Lit n   == Lit n’      =   n == n’
%    e :+: f == e’ :+: f’   =   e == e’ && f == f’
%    e :*: f == e’ :*: f’   =   e == e’ && f == f’
%    _       == _           =   False
%
%  instance Ord Exp where
%    Lit n   <= Lit n’      =   n < n’
%    Lit n   <= e’ :+: f’   =   True
%    Lit n   <= e’ :*: f’   =   True
%    e :+: f <= e’ :+: f’   =   e < e’ || (e == e’ && f <= f’)
%    e :+: f <= e’ :*: f’   =   True
%    e :*: f <= e’ :*: f’   =   e < e’ || (e == e’ && f <= f’)
%    _       <= _           =   False
%
%  instance Show Exp where
%    show (Lit n)   = "Lit " ++ showN n
%    show (e :+: f) = "(" ++ show e ++ ":+:" ++ show f ++ ")"
%    show (e :*: f) = "(" ++ show e ++ ":*:" ++ show f ++ ")"
%Deriving Expressions
%  data   Exp   = Lit Int
%               | Exp :+: Exp
%               | Exp :*: Exp
%               deriving (Eq, Ord, Show)
%
%
%
%
%         Haskell uses types to write code for you!
%

\end{document}



