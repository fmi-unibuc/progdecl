\documentclass[xcolor=pdftex,romanian,colorlinks]{beamer}

\usepackage{../tslides}
\usepackage[all]{xy}
\lstset{language=Haskell}

\title[PD---Introducere]{Programare declarativă\thanks{bazat pe cursul \emph{Informatics 1: Functional Programming} de la \emph{University of Edinburgh}}}
\subtitle{Introducere în programarea funcțională folosind Haskell}
\begin{document}
\begin{frame}
  \titlepage
\end{frame}

\section{Motivație}
\begin{frame}{De ce programare funcțională? De ce Haskell?}
\begin{itemize}
\item E bine să știm cât mai multe limbaje de programare
\item Programarea funcțională e din ce în ce mai importantă în industrie
\begin{itemize}
\item Haskell e folosit în proiecte de Facebook, Google, Microsoft, \ldots
\item mai multe la \url{https://wiki.haskell.org/Haskell_in_industry}
\end{itemize} 
\item Programare funcțională în limbajul vostru preferat de programare:

\begin{itemize}
\item  Java 8, C++x11, C\#, Python, PHP, JavaScript 
\item Funcții anonime ($\lambda$-abstracții)
\item Funcții de procesare a fluxurilor de date: filter, map, reduce
\end{itemize}
\end{itemize}

\begin{block}{De ce Haskell? (din cartea Real World Haskell)}
 [It] is a deep language and [...] learning it is a hugely rewarding experience.
\end{block}
\begin{description}
\item[Nou] Radical diferit de limbajele cu care suntem obișnuiți
\vitem[Puternic] Cod concis, rapid și sigur
\vitem[Plăcut] Tehnici elegante pentru rezolvarea de probleme concrete
\end{description}
\end{frame}

\begin{frame}{Nou}
\begin{block}{Programarea funcțională}
O cale profund diferită de a concepe ideea de software
\end{block}
\begin{itemize}
\item În loc să modificăm datele existente, calculăm valori noi din valorile existente, folosind funcții
\vitem Funcțiile sunt \structure{pure}: aceleași rezultate pentru aceleași intrări
\vitem Distincție clară între părțile pure și cele care comunică cu mediul extern
\vitem Haskell e leneș: orice calcul e amânat cât de mult posibil
\begin{itemize}
\item Schimbă modul de concepere al programelor
\item Permite lucrul cu colecții potențial infinite de date precum [1..]
\end{itemize}
\end{itemize}
\end{frame}

\begin{frame}[fragile]{Puternic}
\begin{itemize}
\item Puritatea asigură consistență

\begin{itemize}
\item O bucată de cod nu poate corupe datele altei bucăți de cod.
\item Mai ușor de testat decât codul care interacționează cu mediul
\end{itemize}

\vitem Evaluarea leneșă poate fi exploatată pentru a reduce timpul de calcul fără a denatura codul
\begin{lstlisting}
firstK k xs = take k (sort xs)
\end{lstlisting}

\vitem Minimalism: mai puțin cod, în mai puțin timp, și cu mai puține defecte
\begin{itemize}
\item
\dots rezolvând totuși problema :-)
\end{itemize}
\end{itemize}
\end{frame}

\begin{frame}{Elegant}
\begin{itemize}
\item Idei abstracte din matematică devin instrumente puternice practice 
\begin{itemize}
\item recursivitate, compunerea de funcții, functori, monade 
\item folosirea lor permite scrierea de cod compact și modular
\end{itemize}
\vitem Rigurozitate:  ne forțează să gândim mai mult înainte, dar ne ajută să scriem cod mai corect și mai curat
\vitem Curbă de învățare în trepte
\begin{itemize}
\item Putem scrie programe mici destul de repede
\item Expertiza în Haskell necesită multă \structure{gândire} și \structure{practică}
\item Descoperirea unei lumi noi poate fi un drum distractiv și provocator

\url{http://wiki.haskell.org/Humor}
\end{itemize}
\end{itemize}
\end{frame}

\begin{section}{Organizare}
\begin{frame}{Plan curs}
\begin{itemize}
\item Partea I: Noțiuni de bază de programare funcțională
\begin{itemize}
\item Funcții, recursie, funcții de ordin înalt, tipuri
\item Operații pe liste: filtrare, transformare, agregare
\end{itemize}
\item Partea II: Noțiuni avansate de programare funcțională
\begin{itemize}
\item Polimorfism, clase de tipuri, modularizare
\item Tipuri de date algebrice - evaluarea expresiilor
\item Operațiuni Intrare/Ieșire
\end{itemize}
\item Partea III: Capitole speciale de Haskell și programare funcțională
\begin{itemize}
\item Agregare pe tipuri algebrice, Monade
\end{itemize}
\end{itemize}
\end{frame}

\begin{frame}{Resurse}
\begin{itemize}
\item Pagina Moodle a cursului: \url{http://moodle.fmi.unibuc.ro/course/view.php?id=449}
\begin{itemize}
\item Prezentările cursurilor, forumuri, resurse electronice
\item Știri legate de curs vor fi postate pe Moodle
\item Notele la teste vor fi postate tot pe Moodle
\item Parola pentru accesarea paginii Moodle: progdecl
\end{itemize}
\vitem \url{http://bit.do/progdecl}
\begin{itemize}
\item Cele mai noi variante ale cursurilor si laboratoarelor.
\end{itemize}
\vitem Cartea online „Learn You a Haskell for Great Good” \url{http://learnyouahaskell.com/}
\vitem Pagina Haskell \url{http://haskell.org}
\begin{itemize}
\item Hoogle \url{https://www.haskell.org/hoogle}
\item Haskell Wiki \url{http://wiki.haskell.org}
\end{itemize}
\end{itemize}
\end{frame}


\begin{frame}{Evaluare}
\begin{block}{Notare}
\begin{itemize}
\item 2 parțiale (par1, par2), examen (ex) și evidențiere la laborator (lab)
\item Nota finală: 1 (oficiu) + par1 + par1 + ex + lab
\item Notele vor fi postate (doar) pe pagina Moodle a cursului.
\end{itemize}
\end{block}

\begin{block}{Condiție de promovabilitate}
\begin{itemize}
\item Nota finală \alert{cel puțin 5}
\begin{itemize}
  \item 5 > 4.99
\end{itemize}
\end{itemize}
\end{block}

\end{frame}

\begin{frame}{Parțial 1}
\begin{itemize}
\item Valorează 4 puncte din nota finală
\item Sâmbăta dintre a 5-a și a 6-a săptămână (4 noiembrie)
\item Pe calculatoare
\item Durată: 1 oră
\item Acoperă materia din Partea I --- Noțiuni de bază
\item Cu acces la materiale descărcate pe calculator
\item Fără acces la rețea/internet
\end{itemize}
\end{frame}

\begin{frame}{Parțial 2}
\begin{itemize}
\item Valorează 3 puncte din nota finală
\item Sâmbăta dinainte de vacanța de iarnă  (16 decembrie)
\item Pe calculatoare
\item Durată: 1 oră
\item Materia din Partea II --- Noțiuni avansate
\item Cu acces la materiale descărcate pe calculator
\item Fără acces la rețea/internet
\end{itemize}
\end{frame}

\begin{frame}{Examen final}
\begin{itemize}
\item Valorează 2 puncte din nota finală
\item În sesiune
\item Materia: toată
\item Durată: 2 ore
\item Cu acces la materiale tipărite
\item Fără acces la rețea/internet
\end{itemize}
\end{frame}

\begin{frame}{Activitate laborator}
\begin{itemize}
\item Se va puncta activitatea în plus față de cerințele obșnuite
\begin{itemize}
\item Probleme suplimentare
\item La alegerea profesorului din tematica laboratorului respectiv
\item Rezolvate în timpul laboratorului
\end{itemize}
\item Maxim un punct pe laborator, la latitudinea profesorului
\item Maxim 10 puncte în total
\item Se adaugă la nota finală cu o pondere de 10\%
\end{itemize}
\end{frame}

\begin{frame}{Observații despre teste și notare}
\begin{itemize}
\item Nu este necesar să dați toate testele pentru promovare
\begin{itemize}
\item ${\it test1} = 7,50 \wedge {\it test2} =  5 \implies {\it final} = 5$ 
\item Dar este necesară nota agregată cel puțin 5
\end{itemize}
\vitem Pentru a lua nota finală 10 sunt necesare toate cele 3 teste
\begin{itemize}
\item ${\it test1} = 10 \wedge {\it test2} =  10 \wedge {\it lab} =  10 \implies {\it final} = 9$ 
\item ${\it test1} = 10 \wedge {\it test2 =  10} \wedge {\it lab =  10} \wedge {\it test3} = 2,50$

\hfill $ \implies {\it final} = 9,50\approx 10$
\item ${\it test1} = 9 \wedge {\it test2 =  9} \wedge {\it lab =  5} \wedge {\it test3} = 9$

\hfill $ \implies {\it final} = 9,50\approx 10$
\end{itemize}
\vitem Cursurile se bazează unele pe altele
\begin{itemize}
\item $\forall j < i . $ Testul $i$ presupune construcții și concepte acoperite de Testul $j$
\end{itemize}
\end{itemize}
\end{frame}


\end{section}

\begin{section}{Programare declarativă vs. imperativă}
\begin{frame}{Programare declarativă vs. imperativă}{Ce vs. cum}
\begin{block}{Programare imperativă (Cum)}
Explic mașinii, pas cu pas, algoritmic, \structure{cum} să facă ceva și ca rezultat, se întâmplă \alert{ce} voiam să se întâmple.
\end{block}
\onslide<2>
\vfill\begin{block}{Programare declarativă (Ce)}
Îi spun mașinii \structure{ce} vreau să se întâmple și o las pe ea să se prindă \alert{cum} să realizeze acest lucru. :-)
\end{block}
\end{frame}

\begin{subsection}{Exemple}
\begin{frame}[fragile]{Operații iterative pe colecții}{\url{http://latentflip.com/imperative-vs-declarative}}
\begin{block}{Imperativ (JS)}
{\small
\begin{lstlisting}[language=JavaScript]
var numbers = [1,2,3,4,5]
var doubled = []
for(var i = 0; i < numbers.length; i++) {
  doubled.push(numbers[i] * 2)
}
\end{lstlisting}}
\end{block}

\onslide<2>
\begin{block}{Declarativ (Haskell)}
{\small
\begin{lstlisting}
numbers = [1,2,3,4,5]
doubled = map (\n -> n * 2) numbers
\end{lstlisting}}
\end{block}
\end{frame}

\begin{frame}[fragile]{Agregarea datelor dintr-o colecție (JS)}{\url{http://latentflip.com/imperative-vs-declarative}}
\begin{block}{Imperativ (JS)}
{\small
\begin{asciijs}
var numbers = [1,2,3,4,5]
var total = 0

for(var i = 0; i < numbers.length; i++) {
  total += numbers[i]
}
\end{asciijs}}
\end{block}

\onslide<2>
\begin{block}{Declarativ (Haskell)}
{\small
\begin{asciihs}
numbers = [1,2,3,4,5]
total = foldl (+) 0 numbers
\end{asciihs}}
\end{block}
\end{frame}

\begin{frame}[fragile]{Extragerea informației din tabele asociate}
{\url{http://latentflip.com/imperative-vs-declarative}}
\begin{block}{Imperativ (JS)}
{\small\vspace{-2ex}
\begin{asciijs}
var dogsWithOwners = []
for(var di=0; di < dogs.length; di++) {
  dog = dogs[di]
  for(var oi=0; oi < owners.length; oi++) {
    owner = owners[oi]
    if (owner && dog.owner_id == owner.id) {
      dogsWithOwners.push({  dog: dog,   owner: owner })
    }
  }}
}
\end{asciijs}}
\end{block}

\onslide<2>
\begin{block}{Declarativ (SQL)}
{\small\vspace{-2ex}
\begin{lstlisting}[language=SQL]
SELECT * FROM dogs INNER JOIN owners
WHERE dogs.owner_id = owners.id
\end{lstlisting}}
\end{block}
\end{frame}


\end{subsection}

\begin{frame}{Programare \alert{imperativă} vs. \structure{declarativă}}{Diferențe}
\begin{itemize}
\item Modelul de computație: \alert{algoritm} vs. \structure{relație}
\vitem Ce exprimă un program: \alert{cum} vs. \structure{ce}
\vitem Variabile/parametrii: atribuire \alert{distructivă} vs. \structure{non-distructivă}
\vitem Structuri de date: \alert{alterabile} vs. \structure{explicite}
\vitem Ordinea de execuție: \alert{efecte laterale} vs. \structure{neimportantă}
\vitem Expresii ca valori: \alert{nu} vs. \structure{da}
\vitem Controlul execuției: responsabilitatea \alert{programatorului} vs \structure{a mașinii}
\end{itemize}
\end{frame}

\end{section}

\begin{section}{Haskell --- privire de ansamblu}
\begin{frame}[fragile]{Expresii și funcții}
\begin{block}{Signatura unei funcții}
\vspace{-2ex}
\begin{asciihs}
fact :: Integer -> Integer
\end{asciihs}
\end{block}
\begin{block}{Definiții folosind if}
\vspace{-2ex}
\begin{asciihs}
fact n = if n == 0 then 1
         else n * fact(n-1)
\end{asciihs}
\end{block}
\vfill \begin{block}{Definiții folosind ecuații}
\vspace{-2ex}
\begin{asciihs}
fact 0 = 1
fact n = n * fact(n-1)
\end{asciihs}
\end{block}
\vfill 
\begin{block}{Definiții folosind cazuri}
\vspace{-2ex}
\begin{asciihs}
fact n
  | n == 0 = 1
  | otherwise = n * fact(n-1)
\end{asciihs}
\end{block}
\end{frame}

\begin{frame}[fragile]{Definiții de liste}
\begin{itemize}
\item Intervale și progresii
\begin{asciihs}
interval = ['c'..'e']       -- ['c','d','e']
progresie = [20,17..1]      -- [20,17,14,11,8,5,2]
progresie' = [2.0,2.5..4.0] -- [2.0,2.5,3.0,3.5,4.0]
progresieInfinita = [3,7..] -- [3,7,11,15,19,..]
\end{asciihs}
\item Definiții prin selecție
\begin{asciihs}
pare :: [Integer] -> [Integer]
pare xs = [x | x<-xs, even x]

pozitiiPare :: [Integer] -> [Integer]
pozitiiPare xs = [i | (i,x) <- [1..] `zip` xs, even x]
\end{asciihs}
\end{itemize}
\end{frame}

\begin{frame}[fragile]{Tipuri. Clase de tipuri. Variabile de tip}
\begin{asciihs}
Prelude> :t 'a'
'a' :: Char
Prelude> :t "ana"
"ana" :: [Char]
Prelude> :t 1
1 :: Num a => a
Prelude> :t [1,2,3]
[1,2,3] :: Num t => [t]
Prelude> :t 3.5
3.5 :: Fractional a => a
Prelude> :t (+)
(+) :: Num a => a -> a -> a
Prelude> :t (+3)
(+3) :: Num a => a -> a
Prelude> :t (3+)
(3+) :: Num a => a -> a
\end{asciihs}
\end{frame}

\begin{frame}[fragile]{Expresii ca valori}{Funcțiile --- „cetățeni de rangul I”} 
\begin{itemize}
\item Funcțiile sunt valori care pot fi luate ca argument sau întoarse ca rezultat
\begin{asciihs}
flip :: (a -> b -> c) -> (b -> a -> c)
flip f = \ x y -> f y x
-- sau alternativ folosind matching
flip f x y = f y x
-- sau flip ca valoare de tip functie
flip = \ f x y -> f y x
-- Currying
flip = \f -> \x -> \y -> f y x 
\end{asciihs}
\item Aplicare parțială a funcțiilor
\begin{asciihs}
injumatateste :: Integral a => a -> a
injumatateste = (`div` 2)
\end{asciihs}
\end{itemize}
\end{frame}

\begin{frame}[fragile]{Funcții de ordin înalt}
{map, filter, foldl, foldr}
\begin{asciihs}
Prelude> map (*3) [1,3,4]
[3,9,12]
Prelude> filter (>=2) [1,3,4]
[3,4]
Prelude> foldr (*) 1  [1,3,4]
12
Prelude> foldl (flip (:)) [] [1,3,4]
[4,3,1]
\end{asciihs}

\begin{block}{Compunere si aplicare}
\begin{asciihs}
Prelude> map (*3) ( filter (<=3) [1,3,4])
[3,9]
Prelude> map (*3) . filter (<=3) $ [1,3,4]
[3,9]
\end{asciihs}
\end{block}
\end{frame}

\begin{frame}[fragile]{Lenevire}
\begin{itemize}
\item Argumentele sunt evaluate doar cand e necesar si doar cat e necesar
\begin{asciihs}
Prelude> head[]
*** Exception: Prelude.head: empty list
Prelude> let x = head []
Prelude> let f a = 5
Prelude> f x
5
Prelude> head [1,head [],3]
1
Prelude> head [head [],3]
*** Exception: Prelude.head: empty list
\end{asciihs}
\item Liste infinite (fluxuri de date)
\begin{asciihs}
ones = [1,1..]
zeros = [0,0..]
both = zip ones zeros
short = take 5 both   -- [(1,0),(1,0),(1,0),(1,0),(1,0)]
\end{asciihs}
\end{itemize}
\end{frame}

\begin{frame}{Intteractiune cu mediul extern}
\begin{itemize}
\item Monade
\item Acțiuni
\item Secvențiere
\end{itemize}
\end{frame}

\end{section}
\end{document}



