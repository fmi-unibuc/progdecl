\documentclass[addpoints,12pt,a4paper]{exam}
%,answers
\usepackage[none]{hyphenat} 
\usepackage{fullpage}
\usepackage{../tdefinition}
\pagestyle{empty}
%\header{Prenumele, numele și grupa: \enspace\hrulefill}{}{Subiect \thequestion}
\extrawidth{0.5in}
\newcommand{\bs}{\char`\\} \newcommand{\bt}{\char`\`}
\newcommand{\us}{\char`\_}

\begin{document}

\begin{center}
Programare declarativă---Examen scris \hfill  12 mai 2017
\end{center}

\pointpoints{punct}{puncte}

\begin{questions}

\question[3\half]{Elemente avansate de programare funcțională.}

Următorul tip de date reprezintă expresiile aritmetice cu o singură variabilă:
\begin{asciihs}
data Expr = X -- variabla
      | Const Integer -- intreg
      | Expr :+: Expr -- adunare
      | Expr :-: Expr -- scadere
      | Expr :*: Expr -- inmultire
      | IfLt Expr Expr Expr Expr -- expresie conditionala
\end{asciihs}
\lstinline$IfLt p q r s$ reprezintă expresia care în Haskell ar fi scrisă ca 
\lstinline$if p < q then r else s$.

\begin{parts}
\part[1\half] Scrieți o funcție \lstinline$eval :: Expr -> Integer -> Integer$ care, dată fiind o expresie și valoarea variabilei \lstinline$X$, calculează valoarea expresiei. De exemplu:
\begin{asciihs}
eval (X :+: (X :*: Const 2)) 3 = 9
eval (X :-: (X :*: Const 3)) 0 = 0
eval (X :-: (X :*: Const 3)) 7 = -14
eval (X :+: X) 2 = 4
eval (Const 15 :+: (Const 7 :*: (X :-: Const 1))) 0 = 8
eval (X :-: (X :+: X)) 4 = -4
\end{asciihs}

\part[2] 
Scrieți o funcție \lstinline$protejeaza :: Expr -> Expr$ that transformă expresia pentru a evita apariția rezultatelor negative, in felul următor:
\begin{itemize}
\item Constantele negative devin 0
\item La toate operațiile de scădere se adaugă o verificare că descăzutul e mai mare decăt scăzătorul; în caz contrar, rezultatul scăderii va deveni 0.
\item Se adaugă o verificare inițială că variabila X este mai mare ca 0; în caz contrar intreaga expresie va fi evaluata la 0
\end{itemize}

Nu simplificați rezultatul prin omiterea testelor care nu par necesare.
De exemplu:
\begin{asciihs}
protect (X :+: (X :*: Const 2))
 = IfLt X 0 (Const 0) 
        (X :+: (X :*: Const 2))
protect (X :-: (X :*: Const 3))
 = IfLt X 0 (Const 0) 
        (IfLt X (X :*: Const 3) (Const 0) (X :-: (X :*: Const 3)))
protect (X :+: X)
 = IfLt X 0 (Const 0) 
        (X :+: X)
protect (Const 15 :+: (Const 7 :*: (X :-: Const 1)))
 = IfLt X 0 (Const 0) 
        (Const 15 :+: (Const 7 :*: IfLt X (Const 1)
                                        (Const 0)
                                        (X :-: Const 1)))
protect (X :-: (X :+: X))
 = IfLt X 0 (Const 0)
        (IfLt X (X :+: X) (Const 0) (X :-: (X :+: X)))
\end{asciihs}
\end{parts}
\question[3\half]
{Elemente de bază de programare funcțională}
\begin{parts}
\part[1]
Scrieți o funcție \lstinline$p :: [Int] -> Int$ care calculează suma cuburilor numerelor pozitive dintr-o listă dată. De exemplu:
\begin{asciihs}
p [-13] = 0
p [] = 0
p [-3,3,1,-3,2,-1] = 36
p [2,6,-3,0,3,-7,2] = 259
p [4,-2,-1,-3] = 64
\end{asciihs}
Folosiți {\em funcții elementare}, {\em descrieri de liste} și {\em funcții din biblioteci}, dar nu {\em recursie}.

\part[1\half]
Scrieți o a doua funcție \lstinline$q :: [String] -> Int$ care se comportă la fel ca \lstinline$p$, folosind doar {\em funcții elementare}, {\em recursie} și {\em funcții din biblioteci}, dar nu  {\em descrieri de liste}.

\part[1]
Scrieți o a treia funcție \lstinline$r :: [String] -> Int$ care se comportă la fel ca \lstinline$p$, de data aceasta folosind următoarele funcții de ordin înalt din biblioteca standard:
\begin{asciihs}
map :: (a -> b) -> [a] -> [b]
filter :: (a -> Bool) -> [a] -> [a]
foldr :: (a -> b -> b) -> b -> [a] -> b
\end{asciihs}
Nu puteți folosi {\em recursie} și {\em descrieri de liste}.
\end{parts}

\question[2]{Elemente specifice limbajului Haskell.}

Se dă următoarea secvență de declarații în Haskell:

\begin{asciihs}
type Id = String
data AExp = Lit Int | Var Id | AExp :+: AExp | AExp :<=: AExp
data Stmt = Decl Id | If AExp Stmt Stmt | Id := AExp | Stmt :; Stmt | Skip
type Mem = [(Id, Int)]
eval :: Stmt -> Mem
\end{asciihs}

Să se completeze definiția funcției \lstinline$eval$ pentru a putea calcula starea memoriei (Mem) în urma execuției instrucțiunii date ca argument.  Observatie: Puteți defini oricâte funcții ajutătoare.

\lstinline$Decl$ inițializează o nouă varibilă în memorie (dacă nu exista deja), cu valoarea 0.
\lstinline$If$ are aceeași semantică ca în C (dacă valoarea expresiei e diferită de 0 se execută prima instrucțiune, dacă nu cea de-a doua). Operatorul \lstinline$:;$ este operatorul de compunere secvențială a instrucțiunilor; semantica lui este: se execută prima instrucțiune mai întâi, apoi a doua instrucțiune se execută în starea de după execuția primei instrucțiuni. \lstinline$Skip$ este instrucțiunea vidă. 

Punctaj: 2 puncte din care 0,5 rezolvarea, + 1 rezolvarea folosind monade + 0,5 rezolvarea folosind monada predefinită cea mai potrivită.
\end{questions}

Se acordă un punct din oficiu.
\end{document}
\question[24] 
Recursivitate, descrieri de liste, map/filter/fold
\begin{parts}
\part[12]
Scrieți o funcție \lstinline$f :: String -> String$ care repetă fiecare caracter succesiv din șirul de intrare odată în plus față de caracterul precedent, începând cu o primă apariție a primului caracter. De exemplu:
\begin{asciihs}
f "abcde"  == "abbcccddddeeeee"
f "ZYw"    == "ZYYwww"
f ""       == ""
f "Inf1FP" == "Innfff1111FFFFFPPPPPP"
\end{asciihs}
Folosiți {\em funcții elementare}, {\em descrieri de liste} și {\em funcții din biblioteci}, dar nu {\em recursie}.
\part[12]
Scrieți o a doua funcție \lstinline$g :: String -> String$ care se comportă la fel ca \lstinline$f$, folosind doar {\em funcții elementare}, {\em recursie} și {\em funcții din biblioteci}, dar nu  {\em descrieri de liste}.
\end{parts}

