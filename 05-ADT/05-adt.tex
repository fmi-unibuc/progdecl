\documentclass[handout,xcolor=pdftex,romanian,colorlinks]{beamer}

\usepackage[export]{adjustbox}
\usepackage{../tslides}
\usepackage[all]{xy}
\usepackage{pgfplots}
\usepackage{flowchart}
\usetikzlibrary{arrows,positioning,calc}
\lstset{language=Haskell}
\lstset{escapeinside={(*@}{@*)}}

\AtBeginSection[]{
  \begin{frame}
  \vfill
  \centering
  \begin{beamercolorbox}[sep=8pt,center,shadow=true,rounded=true]{title}
    \usebeamerfont{title}\insertsectionhead\par%
  \end{beamercolorbox}
  \vfill
  \end{frame}
}


\title[PD---Tipuri date algebrice]{Programare declarativă\thanks{bazat pe cursul \emph{Informatics 1: Functional Programming} de la \emph{University of Edinburgh}}}
\subtitle{Tipuri de date algebrice}
\begin{document}
\begin{frame}
  \titlepage
\end{frame}

\begin{frame}[fragile]{Ce este un tip de date algebric?}

\onslide<2>
\begin{block}{Orice!}
\begin{asciihs}
data Bool = False | True
data Season = Winter | Spring | Summer | Fall
data Shape = Circle Float | Rectangle Float Float
data List a = Nil | Cons a (List a)
data Nat = Zero | Succ Nat
data Exp = Lit Int | Add Exp Exp | Mul Exp Exp
data Tree a = Empty | Leaf a | Branch (Tree a) (Tree a)
data Maybe a = Nothing | Just a
data Pair a b = Pair a b
data Either a b = Left a | Right b
\end{asciihs}
\end{block}
\end{frame}

\subsection{Tipul de date Boolean}

\begin{frame}[fragile]{Tipul de date Boolean}
\begin{asciihs}
data Bool = False | True

not :: Bool -> Bool
not False  =  True
not True   =  False

(&&) :: Bool -> Bool -> Bool
False && q   =  False
True  && q   =  q

(||) :: Bool -> Bool -> Bool
False || q   =  q
True  || q   =  True
\end{asciihs}
\end{frame}


\begin{frame}[fragile]{Definirea egalității și a reprezentării}
{Eq și Show}
\begin{block}{Eq}
\begin{asciihs}
eqBool :: Bool -> Bool -> Bool
\end{asciihs}
\onslide<2>
\vspace{-2ex}
\begin{asciihs}
eqBool False False = True
eqBool True  True  = True
eqBool _     _     = False
\end{asciihs}
\end{block}

\onslide<1->
\begin{block}{Show}
\begin{asciihs}
showBool :: Bool -> String
\end{asciihs}
\onslide<2>
\vspace{-2ex}
\begin{asciihs}
showBool False = "False"
showBool True  = "True"
\end{asciihs}
\end{block}
\end{frame}

\subsection{Anotimpuri}

\begin{frame}[fragile]{Anotimpuri}
\begin{asciihs}
data Season = Spring | Summer | Autumn | Winter

next :: Season -> Season
next Spring = Summer
next Summer = Autumn
next Autumn = Winter
next Winter = Spring
\end{asciihs}
\end{frame}

\begin{frame}[fragile]{Definirea egalității și a reprezentării}
{Eq și Show}
\begin{asciihs}
eqSeason :: Season -> Season -> Bool
eqSeason Spring Spring = True
eqSeason Summer Summer = True
eqSeason Autumn Autumn = True
eqSeason Winter Winter = True
eqSeason _      _      = False


showSeason :: Season -> String
showSeason Spring = "Spring"
showSeason Summer = "Summer"
showSeason Autumn = "Autumn"
showSeason Winter = "Winter"
\end{asciihs}
\end{frame}


\begin{frame}[fragile]{Enumerări și indici}
\begin{asciihs}
  data Season = Winter | Spring | Summer | Fall
  toInt :: Season -> Int
  toInt Winter = 0
  toInt Spring = 1
  toInt Summer = 2
  toInt Fall   = 3

  fromInt :: Int -> Season
  fromInt 0 = Winter
  fromInt 1 = Spring
  fromInt 2 = Summer
  fromInt 3 = Fall

  next :: Season -> Season
\end{asciihs}
\onslide<2>
\vspace{-2ex}
\begin{asciihs}
  next x = fromInt ((toInt x + 1) `mod` 4)

\end{asciihs}
\onslide<1->
\begin{asciihs}
  eqSeason :: Season -> Season -> Bool
\end{asciihs}
\onslide<2>
\vspace{-2ex}
\begin{asciihs}
  eqSeason x y = (toInt x == toInt y)

\end{asciihs}
\end{frame}

\subsection{Forme geometrice}

\begin{frame}[fragile]{Cercuri și dreptunghiuri}
\begin{asciihs}
  type   Radius   =   Float
  type   Width    =   Float
  type   Height   =   Float

  data   Shape    =   Circle Radius
                  |   Rect Width Height

  area :: Shape -> Float
  area (Circle r) = pi * r^2
  area (Rect w h) = w * h
\end{asciihs}
\end{frame}


\begin{frame}[fragile]{Definirea egalității și a reprezentării}
{Eq și Show}
\begin{asciihs}
  eqShape :: Shape -> Shape -> Bool
\end{asciihs}
\onslide<2>
\vspace{-2ex}
\begin{asciihs}
  eqShape (Circle r) (Circle r')  = (r == r')
  eqShape (Rect w h) (Rect w' h') = (w == w') && (h == h')
  eqShape _          _            = False

\end{asciihs}
\onslide<1->
\begin{asciihs}
  showShape :: Shape -> String
\end{asciihs}
\onslide<2>
\vspace{-2ex}
\begin{asciihs}
  showShape (Circle r) = "Circle " ++ showF r
  showShape (Rect w h) = "Rect " ++ showF w 
      ++ " " ++ showF h

  showF :: Float -> String
  showF x | x >= 0     = show x
          | otherwise = "(" ++ show x ++ ")"
\end{asciihs}
\end{frame}

\begin{frame}[fragile]{Teste și operatori de proiecție}
\begin{asciihs}
  isCircle :: Shape -> Bool
  isCircle (Circle r) = True
  isCircle _          = False

  isRect :: Shape -> Bool
  isRect (Rect w h) = True
  isRect _          = False

  radius :: Shape -> Float
  radius (Circle r) = r

  width :: Shape -> Float
  width (Rect w h) = w

  height :: Shape -> Float
  height (Rect w h) = h
\end{asciihs}
\end{frame}

\begin{frame}[fragile]{Pattern-matching}
\begin{asciihs}
  area :: Shape -> Float
  area (Circle r) = pi * r^2
  area (Rect w h) = w * h

  area :: Shape -> Float
  area s =
    if isCircle s then
       let
           r = radius s
       in
           pi * r^2
    else if isRect s then
       let
           w = width s
           h = height s
       in
           w * h
    else error "impossible"
\end{asciihs}
\end{frame}

\subsection{Liste}

\begin{frame}[fragile]{Pattern-matching}
\begin{block}{Declarație ca tip de date algebric}
\begin{asciihs}
   data  List a  = Nil
                 | Cons a (List a)

   append :: List a -> List a -> List a
   append Nil ys          = ys
   append (Cons x xs) ys = Cons x (append xs ys)
\end{asciihs}
\end{block}
\end{frame}

\begin{frame}[fragile]{Constructori simboluri}

\begin{block}{Declarație ca tip de date algebric cu simboluri}
\begin{asciihs}
data  List a  = Nil
              | a ::: List a
  deriving (Show)
infixr 5 :::

(+++) :: List a -> List a -> List a
infixr 5 +++
Nil +++ ys        = ys
(x ::: xs) +++ ys = x ::: (xs +++ ys)
\end{asciihs}
\end{block}

\begin{block}{Comparați cu versiunea folosind notația predefinită}
\begin{asciihs}
   (++) :: [a] -> [a] -> [a]
   [] ++ ys      = ys
   (x:xs) ++ ys = x : (xs ++ ys)
\end{asciihs}
\end{block}
\end{frame}

\begin{frame}[fragile]{Definirea egalității și a reprezentării}
{Eq și Show}
\begin{asciihs}
  eqList :: Eq a => List a -> List a -> Bool
  eqList Nil Nil  = True
  eqList (x ::: xs)  (y:::ys) = x == y && eqList xs ys
  eqList _          _            = False

  showList :: Show a => List a -> String
  showList Nil = "Nil"
  showList (x ::: xs) = show x ++ " ::: " ++ showList xs 
\end{asciihs}
\end{frame}



\subsection{Numere naturale}


\begin{frame}[fragile]{Numerele Naturale (Peano)}
\begin{block}{Declarație ca tip de date algebric}
\begin{asciihs}
   data     Nat   =      Zero
                  |      Succ Nat

   (^^^) :: Float -> Nat -> Float
   x ^^^ Zero      = 1.0
   x ^^^ (Succ n) = x * x ^^^ n
\end{asciihs}
\end{block}
\begin{block}{Comparați cu versiunea folosind notația predefinită}
\begin{asciihs}
   (^^) :: Float -> Int -> Float
   x ^^ 0 = 1.0
   x ^^ n = x * (x ^^ (n-1))
\end{asciihs}
\end{block}
\end{frame}

\begin{frame}[fragile]{Adunare și înmulțire}
\begin{block}{Definiție pe tipul de date algebric}
\vspace{-2ex}
\begin{asciihs}
   (+++) :: Nat -> Nat -> Nat
   m +++ Zero     = m
   m +++ (Succ n) = Succ (m +++ n)

   (***) :: Nat -> Nat -> Nat
   m *** Zero     = Zero
   m *** (Succ n) = (m *** n) +++ m
\end{asciihs}
\end{block}
\begin{block}{Comparați cu versiunea folosind notația predefinită}
\vspace{-2ex}
\begin{asciihs}
   (+) :: Int -> Int -> Int
   m + 0 = m
   m + n = (m + (n-1)) + 1

   (*) :: Int -> Int -> Int
   m * 0 = 0
   m * n = (m * (n-1)) + m
\end{asciihs}
\end{block}
\end{frame}

\begin{frame}[fragile]{Date personale}
\begin{asciihs}
type FirstName  = String  
type LastName   = String  
type Age        = Int  
type Height     = Float  
type PhoneNumber = String  
type Flavor      = String  
                     
data Person = Person FirstName LastName Age Height
    PhoneNumber Flavor


\end{asciihs}

\end{frame}

\begin{frame}[fragile]{Proiecții}
\begin{asciihs}
    firstName :: Person -> String  
    firstName (Person firstname _ _ _ _ _) = firstname  
      
    lastName :: Person -> String  
    lastName (Person _ lastname _ _ _ _) = lastname  
      
    age :: Person -> Int  
    age (Person _ _ age _ _ _) = age  
      
    height :: Person -> Float  
    height (Person _ _ _ height _ _) = height  
      
    phoneNumber :: Person -> String  
    phoneNumber (Person _ _ _ _ number _) = number  
      
    flavor :: Person -> String  
    flavor (Person _ _ _ _ _ flavor) = flavor  
\end{asciihs}
\end{frame}

\begin{frame}[fragile]{Utilizare}
\begin{asciihs}
Main*> let ionel = Person "Ion" "Ionescu" 20 175.2 
    "0712334567" "Caramel"
Main*> firstName ionel
"Ion"  
Main*> height ionel
175.2  
Main*> flavor ionel  
"Caramel"  
\end{asciihs}
\end{frame}

\subsection{Înregistrări}

\begin{frame}[fragile]{Date personale ca înregistrări}
\begin{asciihs}
data Person = Person { firstName :: String  
                     , lastName :: String  
                     , age :: Int  
                     , height :: Float  
                     , phoneNumber :: String  
                     , flavor :: String  
                     }
\end{asciihs}
\end{frame}

\begin{frame}[fragile]{Utilizare}
\begin{itemize}
\item Putem folosi atât forma algebrică cât și cea de înregistrare
\begin{asciihs}
ionel = Person "Ion" "Ionescu" 20 175.2 
    "0712334567" "Caramel"

gigel = Person { firstName = "Gheorghe"
               , lastName="Georgescu"
               , age = 30, height = 192.3
               , phoneNumber = "0798765432"
               , flavor = "Vanilie" }
\end{asciihs}
\item Putem folosi și pattern-matching
\item Proiecțiile sunt definite automat; sintaxă specializată pentru actualizări%
\begin{asciihs}
nextYear :: Person -> Person
nextYear person = person { age = age person + 1 }
\end{asciihs}
\end{itemize}
\end{frame}

\begin{frame}{De ce algebric?}
\end{frame}

\end{document}



