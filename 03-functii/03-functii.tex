\documentclass[xcolor=pdftex,romanian,colorlinks]{beamer}

\usepackage[export]{adjustbox}
\usepackage{../tslides}
\usepackage[all]{xy}
\usepackage{pgfplots}
\usepackage{flowchart}
\usetikzlibrary{arrows,positioning,calc}
\lstset{language=Haskell}


\AtBeginSection[]{
  \begin{frame}
  \vfill
  \centering
  \begin{beamercolorbox}[sep=8pt,center,shadow=true,rounded=true]{title}
    \usebeamerfont{title}\insertsectionhead\par%
  \end{beamercolorbox}
  \vfill
  \end{frame}
}


\title[PD---Funcții]{Programare declarativă\thanks{bazat pe cursul \emph{Informatics 1: Functional Programming} de la \emph{University of Edinburgh}}}
\subtitle{Operatori, Funcții (din nou), Recursie (din nou)}
\begin{document}
\begin{frame}
  \titlepage
\end{frame}

\section{Operatori}
\begin{frame}[fragile]{Operatorii sunt funcții}{Operatori Booleeni}
\begin{asciihs}
not' :: Bool -> Bool
\end{asciihs}
\onslide<2->
\vspace{-2ex}
\begin{asciihs}
not' True = False
not' False = True
\end{asciihs}
\onslide<3->
\begin{asciihs}
(&&&) :: Bool -> Bool -> Bool
\end{asciihs}
\onslide<4->
\vspace{-2ex}
\begin{asciihs}
True &&& b = b
False &&& _ = False
\end{asciihs}
\end{frame}

\begin{frame}[fragile]{Funcțiile sunt operatori}{Operatori aritmetici}
\begin{asciihs}
Prelude> mod 5 2
1
Prelude> 5 `mod` 2
1
\end{asciihs}
\onslide<2->
\begin{asciihs}
divide :: Int -> Int -> Bool
\end{asciihs}
\vspace{-2ex}
\onslide<3->
\begin{asciihs}
x `divide` y = y `mod` x == 0
\end{asciihs}
\onslide<4->
\begin{asciihs}
apartine :: Int -> [Int] -> Bool
\end{asciihs}
\vspace{-2ex}
\onslide<5->
\begin{asciihs}
x `apartine` []       = False
x `apartine` (y:xs)   = x == y || (x `apartine` xs) 
\end{asciihs}

\end{frame}

\begin{frame}[fragile]
{Precedență și asociativitate}
\begin{asciihs}
Prelude> 3+5*4:[6]++8-2+3:[2]==[23,6,9,2]||True==False
\end{asciihs}
\vspace{-2ex}
\onslide<2->
\begin{asciihs}
True
\end{asciihs}

\onslide<3->
\hspace*{-1ex}\begin{tabular}{|l|l|l|l|}
\hline
Precedence & Left associative &	Non-associative &	Right associative\\
\hline
9&	\lstinline$!!$&	&	\lstinline$.$
\\\hline
8&	&	&	\lstinline$^$, \lstinline$^^$, \lstinline$**$
\\\hline
7&	\lstinline$*$, \lstinline$/$, \lstinline$`div`$, \lstinline$`mod`$, &&
\\
&\lstinline$`rem`$, \lstinline$`quot`$		&&
\\\hline
6&	\lstinline$+$, \lstinline$-$ &&
\\\hline
5&	&	&	\lstinline$:$, \lstinline$++$
\\\hline
4&	&	\lstinline$==$, \lstinline$/=$, \lstinline$<$, \lstinline$<=$, \lstinline$>$, \lstinline$>=$,&
\\
&& \lstinline$`elem`$, \lstinline$`notElem`$	&
\\\hline
3&	&	&	\lstinline$&&$
\\\hline
2&	&	&	\lstinline$||$
\\\hline
1&	\lstinline$>>$, \lstinline$>>=$	&	&
\\\hline
0&	&	&	\lstinline|$|, \lstinline|$!|, \lstinline$`seq`$
\\\hline
\end{tabular}
\end{frame}

\begin{frame}
[fragile]
{Declararea precedenței și a modului de grupare}
{\lstinline$infix$, \lstinline$infixl$, \lstinline$infixr$} 
\onslide<3->\ 
{\color<3>{blue}%
\begin{asciihs}
infixl 6 <+>
\end{asciihs}}
\onslide<1->
\vspace*{-2ex}%
\begin{asciihs}
(<+>) :: Int -> Int -> Int
x <+> y = x + y + 1
\end{asciihs}
\begin{asciihs}
*Main> 1 <+> 2 * 3 <+> 4
\end{asciihs}

\onslide<2->
\vspace{-1ex}
\alt<3->{{\color<3>{blue}
\lstinline$13$}
}{
\lstinline$32$\hfill\lstinline$-- (1 <+> 2) * (3 <+> 4)$

\alert{Precedența implicită este 9 (maximă)}
}
\vfill
\onslide<6->\ 
{\color<6>{blue}%
\begin{asciihs}
infix 4 `egal`
\end{asciihs}}
\onslide<4->
\vspace*{-2ex}%
\begin{asciihs}
egal :: Float -> Float -> Bool
x `egal` y = abs(x - y) <= 0.001
\end{asciihs}
\begin{asciihs}
*Main> 1 / 32 `egal` 1 / 33 
\end{asciihs}

\onslide<5->
\vspace{-1ex}
\alt<6->{
{\color<6>{blue}
\lstinline$True$}
}{
Eroare de sintaxă \hfill \lstinline$-- (1 / (32 `egal` 1)) / 33$
}

\vfill\ 
\end{frame}


\begin{frame}<handout:0>[fragile]
{Precedență și asociativitate}
\hspace*{-1ex}\begin{tabular}{|l|l|l|l|}
\hline
Precedence & Left associative &	Non-associative &	Right associative\\
\hline
9&	\lstinline$!!$&	&	\lstinline$.$
\\\hline
8&	&	&	\lstinline$^$, \lstinline$^^$, \lstinline$**$
\\\hline
7&	\lstinline$*$, \lstinline$/$, \lstinline$`div`$, \lstinline$`mod`$, &&
\\
&\lstinline$`rem`$, \lstinline$`quot`$		&&
\\\hline
6&	\lstinline$+$, \lstinline$-$ &&
\\\hline
5&	&	&	\lstinline$:$, \lstinline$++$
\\\hline
4&	&	\lstinline$==$, \lstinline$/=$, \lstinline$<$, \lstinline$<=$, \lstinline$>$, \lstinline$>=$,&
\\
&& \lstinline$`elem`$, \lstinline$`notElem`$	&
\\\hline
3&	&	&	\lstinline$&&$
\\\hline
2&	&	&	\lstinline$||$
\\\hline
1&	\lstinline$>>$, \lstinline$>>=$	&	&
\\\hline
0&	&	&	\lstinline|$|, \lstinline|$!|, \lstinline$`seq`$
\\\hline
\end{tabular}
\end{frame}

\begin{frame}[fragile]
{De ce?}

\begin{block}
{De ce este operatorul \alert{-} asociativ la stanga?}
\onslide<2->
\lstinline$5 - 2 - 1 == (5 - 2) - 1$\hfill \lstinline$--$ \hfill \lstinline$/= 5 - (2 - 1)$
\end{block}

\onslide<3->
\begin{block}
{De ce este operatorul \alert{:} asociativ la dreapta?}
\onslide<4->
\lstinline$5 : 2 : [] == 5 : (2 : [])$
\end{block}

\onslide<6->
\begin{block}
{Care este complexitatea aplicării operatorului \alert{++}?}
\vspace{-2ex}
\begin{asciihs}
(++) :: [a] -> [a] -> [a]
[] ++ ys = ys
(x:xs) ++ ys = x:(xs ++ ys)
\end{asciihs}
\onslide<7->
liniară în lungimea primului argument
\end{block}

\onslide<5->
\begin{block}
{De ce este operatorul \alert{++} asociativ la dreapta?}
\onslide<8->
Vrem ca lungimea primului argument să fie cât mai mică 
\lstinline$l1 ++ l2 ++ l3 ++ l4 ++ l5 == l1 ++ (l2 ++ (l3 ++ (l4 ++ l5)))$
\end{block}
\end{frame}


\section{Recursie (din nou)}
\subsection{Progresii aritmetice}
\begin{frame}[fragile]{Generarea \lstinline$[m..n]$}
\begin{asciihs}
Prelude> [3..7]
[3,4,5,6,7]
Prelude> enumFromTo 3 7
[3,4,5,6,7]
\end{asciihs}

\lstinline$[m..n]$ este o notație pentru  \lstinline$enumFromTo m n$
\begin{asciihs}
enumFromTo :: Integer -> Integer -> [Integer]
\end{asciihs}
\onslide<2>
\vspace{-2ex}
\begin{asciihs}
enumFromTo m n | m > n     = []
               | otherwise = m : enumFromTo (m + 1) n
\end{asciihs}
\end{frame}

\begin{frame}[fragile]{Generarea \lstinline$[m..]$}

\lstinline$[m..]$ este o notație pentru  \lstinline$enumFrom m$
\begin{asciihs}
enumFrom :: Integer -> [Integer]
\end{asciihs}
\onslide<2>
\vspace{-2ex}
\begin{asciihs}
enumFrom m = m : enumFrom (m + 1)
\end{asciihs}

\begin{block}{Exemplu de rulare}
\begin{asciihs}
enumFrom 4 
  = 4 : enumFrom 5
  = 4 : 5 : enumFrom 6
  = 4 : 5 : 6 : enumFrom 7
  = 4 : 5 : 6 : 7 : enumFrom 8 
  = .......
\end{asciihs}
\end{block}
\end{frame}

\subsection{zip și search}
\begin{frame}[fragile]{Zip}
Zip împerechează (în ordine, câte două) elementele a două liste 
\begin{asciihs}
zip :: [a] -> [b] -> [(a,b)]
\end{asciihs}
\onslide<2->
\vspace{-2ex}
\begin{asciihs}
zip [] ys = []
zip xs [] = []
zip (x:xs) (y:ys) = (x, y) : zip xs ys
\end{asciihs}

\begin{block}{Exemplu de rulare}
\begin{asciihs}
zip [0,1,2] "abc"
  = (0,'a') : zip [1,2] "bc"
  = (0,'a') : ((1,'b') : zip [2] "c")
  = (0,'a') : ((1,'b') : ((2,'c') : zip [] ""))
  = (0,'a') : ((1,'b') : ((2,'c') : []))
  = [(0,'a'),(1,'b'),(2,'c')]
\end{asciihs}
\end{block}
\end{frame}

\begin{frame}[fragile]{Zip cu liste infinite}
\begin{asciihs}
zip :: [a] -> [b] -> [(a,b)]
zip [] ys = []
zip xs [] = []
zip (x:xs) (y:ys) = (x, y) : zip xs ys
\end{asciihs}

\begin{block}{Exemplu de rulare (leneșă)}
\begin{asciihs}
zip [0..] "abc"
  = zip (0:[1..]) "abc"
  = zip (0:[1..]) ('a':"bc")
  = (0,'a') : zip [1..] "bc"
  = (0,'a') : ((1,'b') : zip [2..] "c")
  = (0,'a') : ((1,'b') : ((2,'c') : zip [3..] ""))
  = (0,'a') : ((1,'b') : ((2,'c') : zip (3:[4..]) ""))
  = (0,'a') : ((1,'b') : ((2,'c') : []))
  = [(0,'a'),(1,'b'),(2,'c')]
\end{asciihs}
\end{block}

\end{frame}

\begin{frame}[fragile]{Produs scalar}
Pentru doi vectori $\overline{a}$ și $\overline{b}$ de aceeași lungime,
produsul scalar este $\displaystyle\sum_i a_i*b_i$
\begin{asciihs}
dot :: Num a => [a] -> [a] -> a
\end{asciihs}
\onslide<2>
\vspace{-2ex}
\begin{asciihs}
dot xs ys = sum [x * y | (x,y) <- xs `zip` ys]
\end{asciihs}

\begin{block}{Exemplu de rulare}
\begin{asciihs}
[1,2,3] `dot` [4,5,6] 
  = sum [x * y | (x,y) <- [1,2,3] `zip` [4,5,6]] 
  = sum [x * y | (x,y) <- [(1,4),(2,5),(3,6)]] 
  = sum [1*4,2*5,3*6] 
  = sum [4,10,18] 
  = 720
\end{asciihs}
\end{block}
\end{frame}

\begin{frame}[fragile]{Search}
search caută toate pozițiile dintr-o listă pe care apare un element dat.
\begin{asciihs}
search :: Eq a => [a] -> a -> [Int]
\end{asciihs}
\onslide<2->
\vspace{-2ex}
\begin{asciihs}
search xs x = [i | (i,y) <- [0..] `zip` xs, y == x]
\end{asciihs}

\begin{block}{Exemplu de rulare}
\begin{asciihs}
search "abac" 'a'
= [i | (i,y) <- [0..] `zip` "abac", y == 'a']
= [i | (i,y) <- [(0,'a'),(1,'b'),(2,'a'),(3,'c')], y == 'a']
= [0|'a' == 'a'] ++ [1 | 'b' == 'a'] ++ [2 | 'a' == 'a'] ++ [3 | 'c' == 'a']
= [0,2]
\end{asciihs}
\end{block}
\end{frame}


\end{document}



